\documentclass[11pt]{article}
\usepackage{graphicx}
\usepackage{amsmath}
\usepackage{hyperref}\usepackage{color}
\usepackage{parskip}
\usepackage{float}
\usepackage{tabularx}
\usepackage{appendix}
\usepackage{rotating} \usepackage{verbatim} \usepackage{lscape}
\usepackage{dcolumn} \usepackage{ctable} \usepackage{amssymb}
\usepackage{longtable}
\usepackage{threeparttable}
\usepackage{pdflscape}
 \usepackage[ style=authoryear,backend=biber]{biblatex}
 \addbibresource{proposal.bib} \defbibheading{bibempty}{}

\title{Exporting/Importing and firm performance: Evidence from India}
\author{Arjun }
%\institution{National Institute of Public Finance and Policy}
\newcommand{\alert}[1]{#1}

\newcommand{\floatintro}[1]{
  
  \vspace*{0.1in}
  
  {\footnotesize
    
    #1
    
  }
  
  \vspace*{0.01in}
}
%Introduce floatintros
\definecolor{red}{rgb}{0.0,0.0,0.0}
\hypersetup{colorlinks,breaklinks,linkcolor=red,urlcolor=red,anchorcolor=red,citecolor=red}
\floatstyle{ruled} 
\restylefloat{table} 
\restylefloat{figure}

\begin{document}
\maketitle


\begin{abstract}
 
\end{abstract}


\newpage
\small

\tableofcontents

\newpage
\section{Introduction}\label{sec:introduction}


There is a large literature stating that, on average exporting firms
are more productive than their non-exporting counterparts. This can be
explained by the self-selection of highly productive into exporting
and the productivity gains by exporting
(learning-by-exporting). Since after entering the export market, the quality
of goods demanded is really high, this automatically means that only
higher productive firms will choose to enter the market. Moreover,
there is a high costs associated with participating in the export
markets,  firms have to make expectations about demand in the export
market and expose themseleves to risk.

Participation in the import market can also lead to productivity
benefits. Since, there are additional costs involved to importing
goods/equipment as well, firms will only take part if it improves
their productivity or helps in establishing relationships and find
information about demand in the foreign market. 

This relationship will help us in clarifying whether trade promotion
policies will help in firms becoming more productive. This is done by
performing a counter-factual experiment, where we decrease the costs
of importing and exporting by 10 \% and see the change in productivity
of firms.  

We test this theory on the dataset of Indian manufacturing firms and
see the relationship between productivity and exporting/importing.   

\section{Literature Reviwe}\label{sec:litrev}


\section{Data and descriptive statistics}\label{sec:data}
We use firm level data from Centre for Monitoring Indian Economy
(CMIE) and restrict ourselves to manufacturing firms since it provides
us with the largest dataset. We have data till 1992 to 2017. 

We use \cite{aw2011}.

\subsection{Productivity measurement}


\subsection{Defining export starter}\label{subsec:starter}


\subsection{Superior exporter performance}


\section{Research Design}\label{sec:design}


\section{Results}\label{sec:methodology}

\subsection{Do more productive firms self-select to become
  exporters?}\label{subsec:selfselect}


\subsection{Do firms learn to export?}


\subsection{Do firms learn by exporting?}


% \textbf{Heterogeneity in learning}

% The above analysis only considers learning as an average treatment
% effect across all matched pairs. But as discussed in
% Section~\ref{sec:litrev}, learning can vary across firms based on
% certain characteristics. In this section we explore if learning is
% heterogenous and what firm characteristics are correlated with high
% learning effects.

% We divide matched pairs into quartiles based on firm characteristics
% in the period before entry (t -1). The three variables we consider
% are age, size of the firm, and productivity level. For the matched
% pairs in each quartile, we study difference in productivity growth
% of the matched pairs.

% Figures~\ref{f:age} to~\ref{f:prod} in the appendix show that for
% quartiles by each firm characteristic, there is no learning by
% exporting at a horizon of one and two years i.e. the difference in
% productivity growth of the matched pairs is not significantly
% different from zero. However, for quartile 2 w.r.t. age and quartile
% 2 w.r.t. size, there is a significant difference in the productivity
% of treated and control at a horizon of three years after
% treatment. This suggests that there might be some long term gains in
% productivity for some specific firms.



\subsection{Do export starters grow significantly after export market
  entry?}


\section{Robustness Tests}\label{sec:robustness}


\subsection{Changing the definition of an export starter}




\subsection{Alternative measures of productivity}

\subsection{Changing the matching methodology}


\subsection{Summarising the robustness checks}


\section{Conclusion and Policy implications}\label{sec:conclusion}

\printbibliography
\newpage 

\newpage
\appendix




% \begin{table}[tp]
% \caption{Export Statistics by Industry in 2007}
% \label{t:meanexpo}
% \small\input{./tables/meanexpofloat.gen}
% \end{table}

% \begin{figure}[tp]
%   \caption{Age}
%   \label{f:age}
%   \begin{footnotesize}
%   \begin{tabular}{cc}
%     Quartile 1 & Quartile 2\\[-1.5ex]
%     \includegraphics[scale=0.27]{./PICS/quartileage1.pdf}&\includegraphics[scale=0.27]{./PICS/quartileage2.pdf}\\[-1.5ex]
%     Quartile 3 & Quartile 4\\[-1.5ex]
%     \includegraphics[scale=0.27]{./PICS/quartileage3.pdf}&\includegraphics[scale=0.27]{./PICS/quartileage4.pdf}\\[-1.5ex]
%   \end{tabular}
% \end{footnotesize}
% \end{figure}

% \begin{figure}[tp]
%   \caption{Size}
%   \label{f:size1}
%  \begin{footnotesize}
%   \begin{tabular}{cc}
%     Quartile 1 & Quartile 2\\[-1.5ex]
%     \includegraphics[scale=0.27]{./PICS/quartilesize1.pdf}&\includegraphics[scale=0.27]{./PICS/quartilesize2.pdf}\\[-1.5ex]
%     Quartile 3 & Quartile 4\\[-1.5ex]
%     \includegraphics[scale=0.27]{./PICS/quartilesize3.pdf}&\includegraphics[scale=0.27]{./PICS/quartilesize4.pdf}\\[-1.5ex]
%   \end{tabular}
%  \end{footnotesize}
% \end{figure}

% \begin{figure}[tp]
%   \caption{Productivity}
%   \label{f:prod}
%  \begin{footnotesize}
%   \begin{tabular}{cc}
%     Quartile 1 & Quartile 2\\[-1.5ex]
%     \includegraphics[scale=0.27]{./PICS/quartileprod1.pdf}&\includegraphics[scale=0.27]{./PICS/quartileprod2.pdf}\\[-1.5ex]
%     Quartile 3 & Quartile 4\\[-1.5ex]
%     \includegraphics[scale=0.27]{./PICS/quartileprod3.pdf}&\includegraphics[scale=0.27]{./PICS/quartileprod4.pdf}\\[-1.5ex]
%   \end{tabular}
%  \end{footnotesize}
% \end{figure}

\end{document}


