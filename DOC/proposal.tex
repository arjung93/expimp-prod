\documentclass[12pt]{article}
\usepackage[font=small,labelfont=bf]{caption}
\usepackage{graphicx}
\usepackage{amsmath}
\usepackage{hyperref}\usepackage{color}
\usepackage{parskip}
\usepackage{float}
\usepackage{tabularx}
\usepackage{appendix}
\usepackage{rotating} \usepackage{verbatim} \usepackage{lscape}
\usepackage{dcolumn} \usepackage{ctable} \usepackage{amssymb}
\usepackage{longtable}
\usepackage{threeparttable}
\usepackage{pdflscape}
\usepackage{cases}
\usepackage{geometry}


 \usepackage[style=authoryear,backend=biber, url=true, maxcitenames=2]{biblatex}
 \addbibresource{new.bib}
\title{Complementarity between Exporting, Importing and Productivity: Evidence from India}
\author{Arjun Gupta (2014592/823125)}
% \institution{Tilburg University}
\newcommand{\alert}[1]{#1}

\newcommand{\floatintro}[1]{
  
  \vspace*{0.1in}
  
  {\footnotesize
    
    #1
    
  }
  
  \vspace*{0.01in}
}
%Introduce floatintros
\definecolor{red}{rgb}{0.0,0.0,0.0}
\hypersetup{colorlinks,breaklinks,linkcolor=red,urlcolor=red,anchorcolor=red,citecolor=red}
\floatstyle{ruled} 
\restylefloat{table} 
\restylefloat{figure}

\begin{document}
\pagenumbering{gobble}

\begin{center}
\includegraphics[width=3cm]{./PICS/tilburg.png}
\bigskip
\bigskip
\bigskip
\bigskip

 \textbf{\Large Complementarity between Exporting, Importing and Productivity:
  Evidence from India}\\
\bigskip
\bigskip
by\\
Arjun Gupta [2014592]\\

\bigskip
\bigskip
\bigskip
\bigskip
 A thesis submitted in partial fulfillment of the requirements for the
 degree of Master of Science in Econometrics and Mathematical\\
 \bigskip
\bigskip
\bigskip
\bigskip
 Tilburg School of Economics and Management\\
 Tilburg University\\
 \bigskip
\bigskip
\bigskip
\bigskip

Supervisors \\
Dr. Christoph Walsh\\
 \bigskip
\bigskip
\bigskip
\bigskip

January 2019

\maketitle
\end{center}

% \begin{center}
%  
% \end{center}

\begin{abstract}
This paper tries to understand the export and import behavior of
firms on three phenomenon: i) Self-selection: Ex-ante selection of higher
productivity firms into export and import participation,
ii) Learning-by-doing: Ex-post productivity benefits of participation
in the export and import market and iii) Cost complementarity between exporting
and importing: Decrease in costs to exporting if firm is importing and
vice-versa. These hypothesis are analysed on panel data of Indian
Manufacturing firms from Prowess, Centre for Monitoring Indian Economy
(CMIE). I find that the Indian manufacturing firms exhibit: i) self-selection of higher productivity
firms into exporting and importing, ii) Learning-by-doing phenomenon
is observed for importing, but it is not observed for exporting and
iii) lagged and contemporaneous cost complementarity between exporting
and importing after controlling for firm-level characteristics.  
\end{abstract}
\bigskip

% \begin{center}
%
% Economics\\ 

% \end{center}
\newpage
\small

\tableofcontents

\newpage
\pagenumbering{arabic}
\section{Introduction}\label{sec:introduction}

Exporting and importing are the two mediums through which firms can
participate in the international trade market. Exporting involves
selling products and services of a firm to buyers in the international
market, whereas importing involves purchasing intermediate goods for
the use of a firm.  

There is a vast empirical literature  that has documented that exporters tend to
out-perform non-exporters  in terms of wages, capital,
productivity.%% Cite as a footnote some papers%%
  \textcite{bernard1999exceptional}  suggest that this can be due
to the following phenomenon:
\begin{itemize}
\item High productive firms are more likely to export (Self-Selection)
\item Export market participation increases productivity (Learning-by-doing)
\end{itemize}

Self-selection (SS) hypothesis states that highly productive firms
select themselves into exporting as  
participation in foreign market is costly. It is accompanied by large
sunk costs
such as costs of establishing a distribution channel,
cost of traversing bureaucracy etc, and increases the variable cost
through higher transportation costs. % This means that there
% are substantial sunk costs to participating in the trade
% market. Therefore, firms which are more productive enter the
% export market. 

Learning-by-doing hypothesis for exporting  states that exporters deal with
tougher competition in international markets as compared to the
domestic market, and therefore must improve their productivity to
remain active. Moreover, participating in the
international market could lead to knowledge flows from international
buyers, which could further lead to gains in productivity.

However, most of the research in this field has been limited to
exports and research on imports has been relatively low. The same hypothesis (self-selection and learning-by-doing) can  apply
to import behavior of firms. Since importing intermediaries also
involves additional
similar costs like  taxes, transport costs, import duties
etc., firms that are also more productive will \textit{self-select}
into importing. Also, since a firm
participating in the import market can have access to higher quality
of intermediary goods, it might lead to productivity benefits
i.e the firms might exhibit \textit{learning-by-doing} phenomenon. 
\textcite{topalova2011trade}  and \textcite{halpern2011imported}
find that improved access to foreign technology can boost
productivity. 

\textcite{muuls2009imports} and \textcite{aristei2013firms} find that
a large proportion on firm that export also participate in the import
market. This might mean that participating in one trade activity might increase the probability of
participating in the other due to \textit{cost complementarity}. This is
because  firms could gain new information about global distribution
channels, or because learning by doing fosters an increase in productivity.   %% GIVE AN EXAMPLE


India liberalised its economy in
1992 which resulted in lower export and import
tariffs, deregulation of markets, reduction of taxes, and greater 
foreign investment. According to \textcite{topalova2011trade}, \textit{the government's trade policy under the Eighth Five-Year Plan (1992-97) ushered
in radical changes to the trade regime by sharply reducing the role of
the import and export control system. The share of products subject to quantitative restrictions
decreased from 87 percent in 1987-88 to 45 percent in 1994-95, and the actual user
condition on imports was discontinued. And since 1997, the decrease in
output and input tariff has been very marginal.}. Therefore,
firm-level data of Indian firms provide a good opportunity to
investigate  productivity gains from participation in
the trade  market, self-selection of high productivity firms into
exporting and importing and 
whether participation in one trade activity complements participation
in the other activity.

In this paper I try to investigate the following phenomenon in the
export and import behavior of Indian Manufacturing firms:
i)Learning-by-doing, ii) Self-selection and sunk cost and iii) Cost
complementarity (Contemporaneous and Lagged). 

The structure of the paper is as follows: section \ref{sec:lit} shows
the main contributions of the literature to this field, section
\ref{sec:data} summarises the data used in this paper, section
\ref{sec:desc} displays the descriptive statistics of the data,
section \ref{sec:anal} displays the empirical analysis and section 
\ref{sec:conclusion} summarises the main findings of the paper. 

Section \ref{sec:data} provides descriptive evidence that firms that
participate in the trade market have higher capital, labor and
productivity than firms that do not participate in the trade
market with the help of density plots and fixed-effects regression.  It also provides evidence of  persistence in trading
behavior by observing the empirical transition probability observed in
the data. 

The empirical analysis is divided into three parts: Section
\ref{sec:lbd} displays the results for the learning-by-doing
phenomenon, section \ref{sec:ss} displays the results for the sunk and
self-selection hypothesis as well as the lagged cost complementarity
phenomenon and section \ref{sec:biprobit} displays the results for the
Contemporaneous cost cost complementarity phenomenon.

  I investigate the learning-by-doing hypothesis by endogenously
accommodating the the decision to export and import in the
productivity evolution process in the productivity estimation
procedures popular in the literature:
\textcite{levinsohn2003estimating} and
\textcite{ackerberg2006structural}. The results suggest the importing
has a positive effect on productivity (3.6\% and 2.7\% for the two
methods respectively) and exporting has no effect on
productivity. This result suggests that firms experience
learning-by-importing and do not experience learning-by-exporting. 

Then, I estimate a dynamic random effects probit model of the discrete
decision to export and import to find that the lagged decision has the
strongest and positive effect on the decision to export and
import respectively. This suggests the presence of the sunk-cost
hypothesis. Moreover, productivity and capital/labor have a positive
effect on the decision to export and import as well. This suggests the
presence of self-selection of higher productivity, capital and labor,  firms into exporting
and importing. 

I add the lagged decision to import on the dynamic random effects
probit model  of the decision to export and vice-versa. The results show that the lagged
decision to import has a positive effect on the probability of
exporting. The same pattern is observed for the effect of lagged
decision to export on the decision to import. This suggests the
presence of lagged cost complementarity between exporting and
importing. 

 Finally, I estimate a dynamic bivariate probit model for the decision to export
and import. The results show the same phenomenon observed in the
dynamic random effects probit model. On top of that, they also indicate the presence
of contemporaneous cost complementarity as the shocks of the decision
to export and import are positively correlated. 

\section{Literature Review}\label{sec:lit}
Research papers related to this topic can be differentiated into three different
categories:
\begin{enumerate}
\item Importing and Productivity 
\item Exporting and Productivity
\item  Complementarity between exporting and importing
\end{enumerate}
There is another section on the major literature contribution towards
productivity estimation. 

\subsection*{Exporting and Productivity}

\textcite{roberts1997decision} is one of the earliest papers that
tests the sunk cost and self-selection hypothesis. They quantify the effect of prior exporting
decision on the current exporting decision and test the sunk cost
hypothesis of these activities.  They  develop a dynamic discrete-
 choice model of exporting behavior that separates the roles of profit heterogeneity
 and sunk entry costs in explaining plants' exporting status and find
 that sunk costs are significant as prior export experience increases
 the probability of exporting by 60 percent.  

The most popular research paper in this literature is
\textcite{bernard1999exceptional}. They test the self-selection and
 learning-by-doing hypothesis of exports on firms. They find that Good
 plants become exporters i.e. learn to export. and find that exporting
 increases the survival probability but it does not contribute towards
 productivity growth.  

\textcite{wagner2007exports} and \textcite{wagner2012international}
conducted a survey of the important research papers in international
trade. They find  there is self-selection of more productive
firms into exporting in most of the literature, however there is mixed evidence for
learning-by-exporting.  


Some papers have tried to estimated the fixed and sunk costs of
exporting by estimating dynamic model. One of the most popular papers
in this field is \textcite{aw2011}. They estimate a dynamic structural model that captures both the behavioral
and technological linkages among R\&D, exporting, and
productivity. They find that relative to a
plant that does neither activity, export market participation raises future productivity
by 1.96 percent, R\&D investment raises it by 4.79 percent, and undertaking both
activities raises it by 5.56 percent. 

In terms of work in this field with Indian firm level data,
\textcite{haidar2012trade} and \textcite{gupta2018exporting} find evidence of
self-selection of more productive firms into exporting, but they do
not find evidence of learning-by-exporting. 

\subsection*{Importing and Productivity}

Most of the research in this field is mostly concerned with exports.
If a firm resorts to importing inputs, it can have access to
higher quality of intermediate goods and therefore
learning-by-importing phenomenon can be observed. 
The arguments in favor of self-selection of more productive firms say
that importing behavior is associated with  fixed and
sunk costs as it involves a search process for
potential foreign suppliers, inspection of goods, negotiation, contract formulation
etc.  

One of the most important papers in this field
is\textcite{topalova2011trade}. They find that the pro-competitive
effects of the import tariffs led firms to become more
productive, the larger impact appears to have come from 
increased access to foreign inputs. The learning-by-doing and
self-selection for importing has  been observed in
\textcite{muuls2009imports} and \textcite{kasahara2008does}. 

\textcite{halpern2011imported} studies effect of imports on productivity by estimating a structural
model of importers in a panel of Hungarian firms. They find that imports have
a significant and large effect on firm productivity, about one-half of which is due
to imperfect substitution between foreign and domestic goods. This
research paper finds evidence of the learning-by-doing phenomenon. 

\subsection*{Complementarity between Exporting and Importing}

Importing  might pave the way for future
exporting by increasing the productivity or by providing
cost-complementarity between the two activities and vice-versa. 
As far as I am aware, there have been two major papers in this field
i.e \textcite{aristei2013firms} and \textcite{kasahara2013productivity}. 

\textcite{aristei2013firms} estimate a bivariate probit model to
understand the two-way relationship between exporting and importing. 
Thy use  firm-level data for a group of 27 Eastern European and 
Central Asian countries from the World Bank Business Environment 
and Enterprise Performance Survey (BEEPS) over the period 2002–2008, 
and after controlling for size (and other firm-level characteristics),
find that firms’ exporting activity does not increase the
probability of importing, while the latter has a positive effect
 on foreign sales (exporting). 

\textcite{kasahara2013productivity} estimate a stochastic model of
exporting and importing that incorporates heterogeneity in transport
costs and estimate export and import complementarity between the two
trade activities. They find that policies which inhibit the
importation of  foreign intermediates can have a large adverse 
effect on the exportation of final goods.  

Some other papers such have also found complementary effects between
exporting and importing. \textcite{vogel2010higher} provide evidence of imported
intermediate inputs in firm productivity and export scope. 
 \textcite{muuls2009imports} find
existence of self-selection and sunk cost behavior in importing and
find an effect of the lagged decision to import on exporting. 
\subsection*{Productivity Estimation}
Productivity estimation in most of the literature is done using
the methods highlighted below:
\textcite{olley1992dynamics},\textcite{levinsohn2003estimating},
\textcite{ackerberg2006structural} and \textcite{wooldridge2009estimating}. 
These papers take inspiration from one another and are based on a
similar methodology. The method
mentioned in \textcite{olley1992dynamics}
have been explained in section \ref{prodest}.

With regards to effect of the decision to export on productivity , \textcite{de2013detecting} highlights the importance of accommodating
the export decision endogenously in the minimisation problem of the productivity
estimation procedure. 

\section{Data}\label{sec:data}
I use annual firm level data for Indian manufacturing firms from
PROWESS\footnote{Prowess is a database of the financial performance of
  companies. Annual Reports of companies, stock exchanges and
  regulators are the principal sources of the data. It delivers data for over 40,000 Indian companies. This includes listed companies, unlisted public companies and private companies of all sizes and ownership groups}, provided by Centre for Monitoring Indian Economy
(CMIE). This dataset has been used in a number of important paper
studying exporter and importer performance in India \footnote{
  \textcite{de2016prices} ,\textcite{topalova2011trade},
  \textcite{gupta2018exporting} and  \textcite{haidar2012trade} are
  some of the papers which use this data set}.
\begin{center}
\input{./TABLES/indicatordescription.gen}
\end{center}
\begin{center}
\begin{figure}
\includegraphics{./TABLES/compnfirms.pdf}
\caption{Number of firms by year}
\label{compnfirms}
\end{figure}
\end{center}
Table \ref{indicator}
shows the variables I use from the database and defines them.
I deflate nominal values with the yearly Wholseale Price
Index (WPI) to obtain real values. I remove firm-year observations
with missing values of any of the key variables and as result, large number of
firms are removed from the dataset. Figure \ref{compnfirms}  shows the composition of
firms by year in the cleaned dataset. Since the number of firms in the
data from 1988 to 2000 are relatively low,
 I restrict the time period of the analysis from 2001 to 2016.  
 

In Prowess, firms
are under no legal obligation to report their finances, and hence it
is difficult to identify entry and exit of firms. Therefore in this
paper, I do not analyse the effect of export and import on
the survival probability of firms. The non-compulsory disclosure also
creates a bias in favour of large and publicly listed firms. However,
since exporters and importers are generally large, the results could
be biased downwards. 

I create three additional variables \textit{Export Value} and
\textit{Import Value}
% , and
% \textit{Domestic Sales}
by adding the following variables from Table 1.  
\begin{enumerate}
\item Export Value: Sum of the exports of goods and services (\textit{sa\_export\_goods $+$ sa\_export\_serv})
\item Import Value: Sum of imports of raw materials, stores and spares,
  finished goods and capital goods (\textit{sa\_import\_rawmat $+$        sa\_import\_stores\_spares
  $+$ sa\_import\_fg            $+$sa\_import\_cap\_goods})
% \item Domestic Sales: Total Sales - Export Sales( \textit{sa\_sales $-$ Export Value})
\end{enumerate}
Using these value variables, I create dummies for trade market
participation as follows:
\begin{itemize}
\item Export Dummy: $d_{it}^{X}=1$ if \textit{Export Value} $> 0$,
  otherwise $0$
\item Import Dummy: $d_{it}^{M}=1$ if \textit{Import Value} $> 0$, otherwise $0$
\end{itemize} 

Finally, I divide firms in the dataset into the following four categories:
\begin{itemize}
\item None: Firms that do not participate in the export and import
  market 
\item Export only: Firms that participate in the export market only
\item Import only: Firms that participate in the import market only
\item Both: Firms that participate in both export and import market
\end{itemize}

Table \ref{comp_table} displays the composition of firms according to their trade market
participation status. The number of firms that do not
participate in the trade market is low, around 20 to 35
\%. Surprisingly, the number of firms that participate in the trade
market is as high as 75\%. Another interesting feature is that the number
of firms that participate only in the import market is higher than the
firms that participate only in the export market. This can mean that
the demand for foreign intermediaries is relatively high. Also, almost 50 \% of
firms in each year participate in both export and import market. There is
a very high proportion of firms that participate in the both the trade market activities relative to the firms that
participate only in one trade market activity. 

\begin{center}
\input{./TABLES/composition.gen}
\end{center}
\section{Descriptive Statistics}\label{sec:desc}
\subsection{Exporter and Importer Performance}
To see the difference in performance of firms that participate in the trade market
versus those that don't, I show the density plots for log of sales, gross fixed assets,
salaries and for the four categories of 
firms trade behavior defined above, in  
Tables \ref{lsales}, \ref{lgfa} and \ref{tab:lsalary}, respectively. 
\newpage
\newgeometry{ top=2cm}
\begin{center}
\begin{table}[H]
\caption{Summary Statistics of Sales (log)}
\label{lsales}
\begin{tabular}{c}
 \includegraphics{./PICS/denslsales.pdf}   \\ 
 \input{./TABLES/sumstatslsales.gen}  \\  
\end{tabular}
\end{table}
\end{center}
\begin{center}
\begin{table}[H]
\caption{Summary Statistics of Gross Fixed Assets (log)}
\label{lgfa}
\begin{tabular}{c}
 \includegraphics{./PICS/denslgfa.pdf}   \\ 
 \input{./TABLES/sumstatslgfa.gen}  \\  
\end{tabular}
\end{table}
\end{center}
\restoregeometry



\begin{center}
\begin{table}[H]
\caption{Summary Statistics of Salary (log)}
\label{tab:lsalary}
\begin{tabular}{c}
 \includegraphics{./PICS/denslsalary.pdf}   \\ 
 \input{./TABLES/sumstatslsalary.gen}  \\  
\end{tabular}
\end{table}
\end{center}

The distribution of firms that participate in the trade market is more skewed towards the right for all the
variables mentioned above. In the case of sales, firms in category
`None' have an average of 5.52 whereas firms in category `Export Only';
and `Import Only' have and average  of 6.86 and
6.82 respectively. Moreover, firms in category `Both' have an average
of  7.74.  On the other hand, the standard deviation in the 4 
cases is very similar, which suggests that the increase in average is
not due to the presence of outliers. The same pattern is observed for
gross fixed assets and salary  

This suggests that firms that participate in the both export and import market have
higher sales, mean gross fixed assets and 
salaries  than firms that participate in
only export and only import.  And firms that participate in either the
export or import market have higher values  than firms that do not
participate in the trade market. This indicates that firms participating
in the trade market has an positive effect on the observable
characteristics of the firm.

The same pattern in observed for expenditure on raw material (Table \ref{tab:lrawmat}) and
expenditure on power and fuel (Table \ref{tab:lfuel}) in the Appendix
section \ref{sec:fuel}
\subsection{Trade Premia}
Trade premia is defined as the difference in
attributes of firms based on their trade participation status. I
estimate the trade premia using the following Fixed Effect (FE) regression:
\begin{equation}
\label{FE}
 X_{it} = \alpha_i + \gamma_{t} + \beta_{1} d_{it}^{X}+ \beta_{2} d_{it}^{M}+
\beta_{3} d_{it}^{X}*d_{it}^{M} + \beta_{4} age_{it} + \epsilon_{it}
\end{equation}
where $X_{it}$ is firm level characteristics such as Sales, Gross
Fixed Assets, Expenditure on Raw Material and Salary, $d_{it}^X$ is
the export dummy,$d_{it}^M$ is
the import  dummy, the interaction term between these two variables
and $age_{it}$ is the age of the firm. I estimate this equation using
time and firm fixed
effects to account for time invariant firm characteristics, and any
time-specific effects.

\begin{center}
\input{./TABLES/expimppremia.gen}
\end{center}

Table \ref{expimppremia} displays the results for the above
regression. The coefficients for both export and import dummy are positive
and significant at 1\% significance level. Column 1 of this table
shows  that firms  that
participate in the export or import market have higher (0.422 or 0.430
respectively) log of sales than
firms that do not participate  trade market activity. The
interaction term value of $-0.060$ is not higher than the coefficient
value of the export and import dummy. Therefore, the cumulative
effect of participating in both activities is higher than the effect
of participating in just one trade activity. 

 The same trend
is also observed for  gross fixed assets, salary and expenditure on
raw materials.  The age variable also
has a positive coefficient and is significant at 1\% significance
level. Therefore, the older a firm becomes the higher its capital,
assets etc. become. 

This section verifies the presence of trade premia in our
dataset. It further substantiates the point that firms that
participate in one trade market activity (export or import) have
better sales, gross fixed assets etc than firms that do not
participate in the trade market. Furthermore, firms that participate
in the both of the trade market activities have a higher trade premia
than the firms that participate in one of the trade activities. 

In the next section, I descriptively investigate whether exports have an effect
on imports and vice-versa.  

\subsection{Complementarity between Exporting and Importing}
The difference between density plots for export value of firms that
lie under category `Both'  versus those that lie in `Export Only' will
give a better understanding of whether there is an effect of one activity on the other. 
Table \ref{tab:lexport} displays the the density plot  of the log
values of export for firms in the category `Export only' and `Both'
defined above. The table shows that firms in category 'both'
 have a higher mean of exports (5.24) than firms in category 'Export
 Only' (3.56). This suggests that importing has a
positive association with exporting such that firms export more if
they also import.

Table \ref{tab:limport} displays the density plot of the log value of
import  for firms in category 'Both' and `Import Only'.
Firms that participate in both the
export and import market have higher imports (4.95) than firms that only
participate in the import market (3.41). Again, this shows that firms
import more if they are also exporters. 
\newgeometry{top=3cm}
\begin{center}
\begin{table}[H]
\caption{Summary statistics of Export (log)}
\label{tab:lexport}
\begin{tabular}{c}
 \includegraphics{./PICS/denslexport.pdf}   \\ 
 \input{./TABLES/sumstatslexport.gen}  \\  
\end{tabular}
\end{table}
\end{center}

\begin{center}
\begin{table}[H]
\caption{Summary statistics of Import (log)}
\label{tab:limport}
\begin{tabular}{c}
 \includegraphics{./PICS/denslimport.pdf}   \\ 
 \input{./TABLES/sumstatslimport.gen}  \\  
\end{tabular}
\end{table}
\end{center}
\restoregeometry

Tables \ref{tab:lexport} and \ref{tab:limport} suggest that both these activities have a
positive effect on the other and this might be because importing
complements exporting and vice-versa. This complementarity needs
further research, and is the focus of my paper. 

I estimate the trade premia similar to the regression in equation
\ref{FE} to see the effect decision to import on exporting value. This is done by estimating the
following fixed-effects (FE) regression:

$$  log(Export)_{it} = \alpha_{i} + \gamma_{t} +  \beta_{1} d_{it}^{M}+
+ \beta_{2} age_{it} + \epsilon_{it}$$

$$  log(Import)_{it} = \alpha_{i} + \gamma_{t} + \beta_{1} d_{it}^{X} + \beta_{2} age_{it} + \epsilon_{it}$$ 

\begin{center}
\input{./TABLES/prodpremia.gen}
\end{center}

The first two columns of table \ref{prodpremia} display these
results. The discrete decision to import has a positive and
significant effect on the value of exports such that the discrete
decision to participate in the import market increases the log value of
export by 0.506. The discrete decision to
export also has a positive and significant effect on the quantity of
imports by increasing the log value of imports by 0.458.  This further suggests the presence of complementarity
between exporting and importing. 

In the next section, I investigate the effect of exporting and importing on
simple measures of productivity.
 
\subsection{Productivity and Export/Import}

\textcite{gupta2018exporting} define a simple measure of productivity known
as \textit{capital productivity}. It is defined as the log of value added per
unit of capital used by a firm:
\begin{equation}
 log(VA_{it}) - log(k_{it})
\end{equation}
where $VA_{it}$ is firm-level value added, computed as total industrial sales plus
change in stock minus power and fuel expenditures, and raw material
expenses. 
 Table \ref{tab:capprod} displays the summary statistics for this variable
based on the trade activity status. The mean of capital
productivity increases as activity status moves from \textit{None} to
\textit{Export only or Import only} to \textit{Both}, whereas the
standard deviation decreases.  
\begin{center}
\begin{table}[H]
\caption{Summary statistics of Capital Productivity (log)}
\label{tab:capprod}
\begin{tabular}{c}
 \includegraphics{./PICS/denscapprod.pdf}   \\ 
 \input{./TABLES/sumstatscapprod.gen}  \\  
\end{tabular}
\end{table}
\end{center}

%% TDDO: Revise
  Table \ref{tab:capprod} displays that firms which participate in
the trade market  have a higher  capital productivity as firms
participating in the trade market have a higher mean than those firms
that do not particii.epate in the trade. 

\begin{center}
\begin{table}[H]
\caption{Summary statistics of Profit to Sales}
\label{tab:pts}
\begin{tabular}{c}
 \includegraphics{./PICS/denspatsales.pdf}   \\ 
 \input{./TABLES/sumstatspatsales.gen}  \\  
\end{tabular}
\end{table}
\end{center}

I also use a measure of firm profitability, calculated by dividing the profit
after tax of
a firm with its sales. Table \ref{tab:pts} shows the summary
statistics for this variable. The same pattern is observed for this variable as well. The mean increases from -0.1 when a firm is not participating in
the trade market to -0.03 and  -0.06 when a firm is participating in
import and export respectively to 0.01 when a firm is participating in
both of the trade market activities.  

The last two columns of table \ref{prodpremia} estimate the trade
premia for the crude measures of the productivity defined above i.e
Capital Productivity and Profit-to-sales ratio. Both
of the crude measures of productivity are positively correlated to the discrete
decisions of importing and exporting. The discrete decision to participate
in the export market increases the capital productivity by 0.159 and
profit to sales ratio by 0.061. The discrete decision of importing
increases the  capital productivity by 0.097 and
profit to sales ratio by 0.021. 
Moreover, the interaction term is lower than the coefficient of
the two decision in column 3 and 
is not significant in column 4. This suggests that participation in both activities
leads to higher productivity than participation in one activity.
 
\subsection{Transition Probability}
%% change title


Table \ref{tab:transition} displays the transition probabilities
observed in the data  and the values in the brackets represent the
number of firms.  There are very high levels of persistence from
\textit{None} in t-1 to \textit{None} in t. Moreover,   high levels of persistence
are also observed in  \textit{Both}  (91.5\%). This means that there must be high sunk
costs to enter in the export and import market since only 12\% of the
firms  that do not participate in the trade market in t-1 start
participating in the trade market in t.

Interestingly, the levels of
persistence in \textit{Import Only} and \textit{Export Only}  is not
as high. A large portion of firms switch to participating in both the trade market
activities in period t. This suggests that participating in export in time
period in t-1 increases the probability of participating in import  markets in time period
t and vice-versa. Also, the number of firms that flip-flop (switch
trade market status) is low, which provides further evidence of sunk costs
to participating in the trade market as well. 

\begin{table}[H]
\begin{center}

\caption{Empirical Transition Probability}
\label{tab:transition}
\input{./TABLES/transition.gen}
\end{center}

\end{table}


This section provides descriptive evidence that firms that participate
in the trade market are bigger, and have higher
productivity than firms that do not participate in the trade
market. It provides evidence that exporting firms export more if they
participate in the import market as well and vice-versa. It also
provides evidence that there is persistence in trade status
(especially when the firms are not participating in the trade market
and when they are participating in the export and import market). 

In the next section, I proceed to calculate productivity using more
sophisticated techniques used in the literature and investigate the endogenous effect
of the decision to export and import on productivity. Then, I investigate if the
sunk cost hypothesis and self-selection hypothesis is observed using a
dynamic random effects probit model. I end the next section by
estimating a dynamic bivariate probit model to examine the complementary
nature of these two activities. 

\section{Empirical Analysis}\label{sec:anal}

The empirical analysis in this section is divided into the  three
parts:
\begin{itemize}
\item Learning-by-Doing: How does lagged choice of export and import
  impact productivity?
\item Self-Selection and Sunk Cost Hypothesis : Selection of more productive firms into
  exporting and importing and persistence of these activites 
\item Complementarity between exporting and importing (Lagged and
  Contemporaneous): Does engaging
  in one activity complement participation in the other?
\end{itemize}

\subsection{Learning-by-doing}\label{sec:lbd}

 I assume that the firms have a Cobb-Douglas
production function: 

\begin{equation}
y_{it} =   \beta_{l}l_{it} + \beta_{k}k_{it} +
\omega_{it} + \eta_{it} 
\end{equation}
where $l_{it}$ is the labor, $k_{it}$ is capital, $\omega_{it}$ is the
total factor productivity or unobserved productivity and $\eta_{it}$
is a unknown shock affecting the firms output. However, the OLS
estimation provides biased results as it does not account for
simultaneity problem i.e the productivity of a firm should be
correlated with the inputs of production. 

There is a vast literature on the estimation of productivity starting
with the seminal paper in productivity estimation: \textcite{olley1992dynamics} and subsequent
modifications of their method: \textcite{levinsohn2003estimating},
\textcite{ackerberg2006structural} and
\textcite{wooldridge2009estimating}. The estimation strategy of
\textcite{olley1992dynamics} is highlighted in section \ref{op}. 

I use the  methods shown in \textcite{levinsohn2003estimating} and
\textcite{ackerberg2006structural} to estimate Cobb-Douglas
parameters. 


\textcite{levinsohn2003estimating} \textbf{LP} uses a  strategy
similar to \textcite{olley1992dynamics} but use intermediate input demand
as the function to invert out $\omega_{it}$. 
Here, the intermediated material demand function is given by:
$$  m_{it} = m_{it}(\omega_{it}, k_{it})$$
This function is assumed to be montonically increasing and therefore
productivity can be found by inverting the function above. Therefore,
we can write the Cobb-Douglas equation  as: 
$$ y_{it} = \beta_{l}l_{it} + \phi_{it}(m_{it},k_{it})$$
where $\phi_{it}(m_{it},k_{it}) =  \beta_{k}k_{it}+ \omega_{it}(m_{it}, k_{it})$
They suggest the  use of a third degree polynomial to approximate 
$\phi_{it}$ and substitute it into the equation above to give the
following result: 

$$  y_{it} =  \beta_{l}l_{it} + \sum_{i=0}^{3} \sum_{j=0}^{3}
\delta_{ij}k_{it}^{i}m_{it}^{j}$$
The coefficient is $\beta_{l}$ is estimtated by OLS using the equation
above as they assume that the labor is a flexible input i.e there are
no labor adjustment costs and $\hat{\phi}_{it}$ is estimtated by
subtracting the effect of labor from
the fitted value of the above equation:
$$ \hat{\phi}_{it} = \hat{y}_{it} - \hat{\beta}_{l}l_{it} =
 \sum_{i=0}^{3} \sum_{j=0}^{3}
\hat{\delta_{ij}}k_{it}^{i}m_{it}^{j}$$
Therefore, 
So, for any value of $\beta_{k}$:
$$\hat{\omega}_{it} = \hat{\phi}_{it} - \beta_{k}k_{it}$$
 It is also assumed that $\omega_{it}$ follows a first order Markov
process : 
$$\omega_{it} = E[\omega_{it}|\omega_{it-1}] + \epsilon_{it}$$
They  assume a polynomial expansion of the expectation above to give:
$$ \omega_{it}=  \gamma_{o}+\gamma_{1}\omega_{it-1} +
\gamma_{2}\omega_{it-1}^2 + \gamma_{3}\omega_{it-1}^3 + \epsilon_{it} $$ 
Therefore, the value of $\beta_{k}$, for which the expression below is
minimized is chosen to be the coefficient of capital.  
\begin{equation}
\beta_{k}^{*} = arg \underset{\beta_{k}}{\min}\sum_{i=1}^{N}\sum_{t=2}^{T} (y_{it} - \hat{\beta_{l}}l_{it} -
\beta_{k}k_{it} - \hat{E[\omega_{it}|\omega_{it-1}]})^2 
\end{equation}

On the other hand, \textcite{ackerberg2006structural} \textbf{ACF} use a
similar strategy but, suggest that labour might be
correlated with productivity. Therefore, they write  the firms input
material demand as a function of productivity, capital as well as
labor : 
$$ m_{it} = f_{it}(\omega_{it}, k_{it}, l_{it})$$
Inverting this function for $\omega_{it}$ and substituting into the
production function results in the following 
equation of the form:
\begin{equation}
y_{it} = \beta_{l}l_{it} + \beta_k k_{it} + f_{it}^{-1}(m_{it},
k_{it}, l_{it})+ \epsilon_{it}
\end{equation}
They suggest that the  labor coefficient along with capital since it is correlated with
productivity. 
They suggest the following steps:
\begin{enumerate}
\item Obtain $\phi_{it}(m_{it}, k_{it}, l_{it}) = \beta_{l}l_{it} + \beta_k k_{it} + f_{it}^{-1}(m_{it},
k_{it}, l_{it})$ by regressing $y_{it}$ on polynomial approximation of
$\phi_{it}(m_{it}, k_{it}, l_{it})$
\item Use the Markovian nature of $\omega_{it} =
  E(\omega_{it}|\omega_{it-1}) + e_{it}$
and use the following moment equations to estimate $\beta_{k}$ and
$\beta_{l}$:
\begin{equation}
E[e_{it}|(k_{it}, l_{it-1})]= 0
\end{equation}
\end{enumerate} 

However, exogenously regressing lagged  export and import variables  on
productivity (residuals of the Cobb-Douglas function) suggests that past
export and import performance  does not impact future revenue and the
capital coefficient will be biased if capital is correlated with the
export or import status.  This has been highlighted by \textcite{de2013detecting}, and
they suggest that the trade activities should be accommodated
endogenously in the productivity evolution process. This is done by
accommodating the the lagged export and import variable into the
minimisation procedure of productivity:

$\omega_{it} =
  E(\omega_{it}|\omega_{it-1}, d_{it-1}^{X}, d_{it-1}^{M}) + e_{it}$

% \subsection{Learning-by-doing}


% I estimate the  endogenous  effect of export/import on productivity
% using two techniques  used widely in this field: 
%  \textcite{levinsohn2003estimating} (LP)and
%  \textcite{ackerberg2006structural} (ACF).
% ACF results are shown to highlight the robustness of the results. 
% \subsubsection{Levinsohn-Petri}
% \textcite{levinsohn2003estimating} assume the following  Cobb-Douglas production function: 
% \begin{equation}
% y_{it} = \beta_{o} + \beta_{l}l_{it} + \beta_{k}K_{it} +
% \omega_{it}(m_{it}, k_{it}) + \eta_{it} 
% \end{equation}

% Using OLS residuals of the Cobb-Douglas estimates provide biased
% estimates of productivity since there is correlation between
% productivity and characteristics  of firms.   

I use the log values of the following variables for the estimation
procedure:  gross fixed assets as a measure of capital, salary
as a measure of labor and a expenditure on raw materials as a measure
of intermediated input. Since productivity evolution is  assumed to
have a Markovian nature , I assume the following form of productivity
evolution:
\begin{equation}
 \omega_{it} = \alpha_{o} + \alpha_{1}\omega_{it-1} +
\alpha_{2}\omega_{it-1}^{2} + \alpha_{3}\omega_{it-1}^{3}+
\alpha_{4}d_{it-1}^{X} + \alpha_{5} d_{it-1}^{M} +
\alpha_{6}d_{it-1}^{X}d_{it-1}^{M}  + \nu_{it}
\end{equation} 
where $d_{it-1}^{X}$ and $d_{it-1}^{M}$ is the discrete decision to
export and import respectively. 

The estimates of the Cobb-Douglas and  the productivity evolution
coefficients  using the Levinsohn-Petrin
method are shown in Table \ref{regLP} and \ref{prod}  respectively. 
\input{./TABLES/regLP.gen}
\input{./TABLES/prod.gen} 
Table \ref{prod} shows that coefficients of $\omega_{it-1},
\omega_{it-1}^{2}$ and $\omega_{it-1}^{3}$ are significant at 1\%. Therefore, productivity evolutions depends strongly
on the past productivity and the coefficients suggest that there is a
non linear relationship with past
productivity. The coefficient of $d_{it-1}^{X}$ is positive but it is
not significantly different than zero and  coefficient of
$d_{it-1}^{M}$  is positive and significant at 1\% level. The
interaction term between exporting and importing also does not have a significant effect on productivity.
The results suggest lagged decision to export does not a have
a significant effect on productivity (firms do not experience learning-by-exporting). However, 
lagged decision to import causes the productivity to increase by 3 per
cent (firms experience learning-by-importing).  


Table \ref{regLPcont} and \ref{prodcont} in the Appendix \ref{apendix:lbd} display the Cobb-Douglas and
productivity evolution
coefficients  when the
productivity evolution is dependent on the lagged continuous log value of export
and import rather than the lagged decision to export and import.
The effect of export and import has the same sign in this case as
well. It shows that lagged
continuous value of export does not have a significant effect on
productivity and  increase in lagged continuous log value of import  by 1 unit
increases the productivity by 3.6 \%.  



As written before, \textcite{de2013detecting} say that exogenously accommodating the
decision to export and import implies that
past export and import experience has no impact on direct technological
improvements. Therefore, the coefficient of capital will be biased
upwards if the decision to participate in the trade market correlated with capital. Table
\ref{regLPex} displays the coefficients when the export and import
decision is not endogenously accommodated in 
productivity evolution. In this case, the coefficient of capital is
biases upwards (0.452 compared to 0.437 in the endogenous case)  since
the variation in output is attributed more towards capital rather than productivity.   
\input{./TABLES/regLPex.gen}

% \subsubsection{\textcite{ackerberg2006structural}}
The results from the estimation method mentioned in
\textcite{ackerberg2006structural} \textbf{ACF} are shown in \ref{regACF} and \ref{prodACF}
. The coefficient of labor is similar in this
estimation method which further suggests that labor is a flexible
input as assumed in the \textbf{LP} estimation procedure. The results also
show strong dependence on past productivity. In terms of the endogenous effect
of export and import on productivity is roughly the same. The export
decision does not have a significant effect on productivity and the
import decision has a 2.6\% increase to the productivity. These
results are similar to the results from the Levinsohn-Petrin
estimation.   
 

\input{./TABLES/regACF.gen}
\input{./TABLES/prodACF.gen}

Tables \ref{regACFcont} and \ref{prodACFcont} in Appendix \ref{apendix:lbd}  display the Cobb-Douglas and
productivity evolution
coefficients,  when the
productivity evolution is dependent on the lagged continuous lag value of
export and importing. The results also show a similar effect of export
and import on productivity as the discrete choice results.  

A drawback of this estimation procedure is that 
the Cobb-Douglas function used in the estimation procedure  does not
account for immediate impact (effect at time t) of 
importing behavior to the output of a firm.  It is assumed that
participating in the import market has an effect only in increasing the
productivity of the firm in the next period (at time t$+$1). 


Based on the results above,  I conclude that manufacturing firms experience learning-by-importing and
do not experience learning-by-exporting. 
In the next section, I investigate if more productive self-select into
exporting/importing and if the sunk cost hypothesis is observed for
the discrete decision to export and import with the help of a dynamic probit
random effects model. 
\subsection{Self Selection and Sunk Cost hypothesis}\label{sec:ss}
The self-selection hypothesis states that entry into the trade market
involves fixed and sunk costs and only the most productive firms can
overcome these trade costs. Therefore, to participate in the
export and import market a firm must pay a certain costs and only the most
productive are able to pay the costs. To investigate this hypothesis, I
estimate a dynamic random effects probit model similar to the model used
in \textcite{roberts1997decision}. I define $d_{it}^{X}$ as the discrete
decision to export, where
 Bellman equation for a profit maximising firm deciding to enter the
 export market is:
\begin{equation}
V_{it}(S_{it})= \underset{d_{it}^{X}}{\max}\:  \mathbb{E}_{t}(\sum_{i=0}^{\infty} \delta^{t+i}R_{i,t+i}|S_{it})
\end{equation}
 where $\delta$ is the one-period discount factor, $S_{it}$ is the
 relevant state variables affecting the firms decision, $R_{ij}$ is
 the revenue. The equation above can also be written as:
\begin{equation}
V_{it}(S_{it})= \underset{d_{it}^{X}}{\max} (\pi^{D}(S_{it}) + d_{it}^{X}(\pi^{X}(S_{it})- f^{X} -
c^{X}(1-d_{it-1}^{X}))  + \sum_{j=t+1}^{\infty} \delta^{t-j}R_{ij}|S_{it})
\end{equation}
where $\pi^{D}(S_{it})$ is the domestic profit, $\pi^{X}(S_{it})$ is the export
profit, $f^{X}$ is the fixed cost of exporting and $c^{X}$ is the sunk
cost of exporting. 
\begin{equation}
V_{it}(S_{it})= \underset{d_{it}^{X}}{\max} \:\mathbb{E}
\Big(\pi^{D}(S_{it}) + d_{it}^{X}(\pi^{X}(S_{it}) - f^{X} -
c^{X}(1-d_{it-1}^{X}))  + \delta \mathbb{E} (V_{it}(S_{it+1})|S_{it},
d_{it}^{X}) \Big)
\end{equation}

 Thus, a firm will participate in the export market if:
\begin{equation}
\begin{aligned}
\pi_{it}^{*} = \pi^{X}(S_{it})  +
\delta \mathbb{E}_{t}(V_{i,t+1}(S_{it+1}|S_{it},d_{it}^{X}=1) -
V_{i,t+1}(S_{it+1}|S_{it},d_{it}^{X}=0))\\ 
-  (f^{X} +
c^{X}(1-d_{it-1}^{X})) \geq 0
\end{aligned}
\end{equation}
It is assumed that the state variables entering the Bellman equation
are the following: $S_{it}= (k_{it}, l_{it}, \omega_{it},
d_{it-1}^{X})$ i.e capital, labor, productivity and lagged decision to
participate in the export market. In the model above, it is assumed
that a firm pays a sunk cost if it did not participate in the export
market in the previous period  ($d_{it-1}^{X} = 0$).  If there are no sunk costs, then
according to the Bellman equation above, a firm will export as long as
the the current profits plus the discounted expected value is greater
than the fixed costs of exporting. 
The reduced form expression of the equation above is: 
\begin{equation}
  d_{it}^{X}=\begin{cases}
   1 , & \text{if $\pi_{it}^{*} \geq 0$}.\\
   0 , & \text{if $\pi_{it}^{*}<  0$}.
  \end{cases}
\end{equation}
To enable to reduced form estimation of the model, I write the equation above as a linear
function of the relevant state variables along with dummy variables to
adjust for industry and time effects, to give the equation below:

\begin{equation}
  \pi_{it}^{*}=   \gamma_{1}^{X} d_{it-1}^{X} + 
\gamma_{3}^{X} \hat{\omega}_{it}  + \beta_{1}^{X}k_{it}  +\beta_{2}^{X}l_{it}+
s_{i}^{X} + \mu_{t}^{X}  + \alpha_{i}+ \epsilon_{it}^{X}
\end{equation}
A positive and significant coefficient of  $d_{it-1}^{X}$ provides
evidence supporting the presence of sunk cost to participate in the
export market  as a positive coefficient means that there is a
persistence in the exporting behavior. 
If the sunk cost hypothesis did not hold then the probability of exporting
should not depend on the lagged decision to export. Moreover, a positive and significant
coefficient of $\omega_{it}$  provides evidence of the presence of 
self-selection hypothesis as firms with high
productivity have a higher probability of participating  in the export
market. Finally, a positive and significant coefficient of $k_{it}$
and $l_{it}$ provides evidence indicating that bigger firms have a
higher chance of participating in the export market. 




A similar model can be estimated for the discrete decision to import,
since importing also involves additional fixed and sunk costs, a firm
would be only participate in the import market if gets productivity
benefits to overcome the costs. Learning-from-importing has been
demonstrated in the previous section.  The Bellman equation of a firm
deciding to enter the import market is the following: 

\begin{equation}
V_{it}(S_{it})=  \underset{d_{it}^{M}}{\max}\Big(\pi(S_{it}) - d_{it}^{M}( f^{M} +
c^{M}(1-d_{it-1}^{M}))  + \delta \mathbb{E} (V_{it}(S_{it+1})|S_{it}, d_{it}^{M})\Big)
\end{equation}
where $\pi$ is total profit of a firm, $f^{M}$ is the fixed cost of
importing and  $c^{M}$ is the sunk cost of importing and  $S_{it} = (k_{it}, l_{it}, \omega_{it},
d_{it-1}^{M})$. 

Since $S_{it}$ contains productivity and the previous section suggests
that firms exhibit productivity improvements. Therefore,  this Bellman equation assumes that
importing in time t provides benefits  in t+1 as the decision to
import increases the expected value productivity in the next period. Here, a firm will participate in the
import market if:
\begin{equation}
\begin{aligned}
\pi^{*}= 
\delta \mathbb{E}_{t}\Big(V_{i,t+1}(S_{it+1}|S_{it},d_{it}^{M}=1) -
V_{i,t+1}(S_{it+1}|S_{it},d_{it}^{X}=0) \Big) \\-  
(f^{M} + c^{M}(1-d_{it-1}^{M})) \geq 0
\end{aligned}
\end{equation}

Therefore, a firm will participate in the import market if the
discounted productivity productivity benefits of participating in the
import market outweigh the costs to participate in the import
market. This can be tested with a reduced form dynamic probit model similar to
the discrete decision to export: 
\begin{equation}
  d_{it}^{M}=\begin{cases}
   1 , & \text{if $\pi^{*} \geq 0 $}.\\
   0 , & \text{otherwise}.
  \end{cases}
\end{equation}
Therefore, the reduced form equations for the \textbf{base model} of both the discrete choices
can be written as:
\begin{equation}
d_{it}^{X*} = \gamma_{1}^{X} d_{it-1}^{X} + 
\gamma_{3}^{X} \hat{\omega}_{it}  + \beta_{1}^{X}k_{it}  +\beta_{2}^{L}l_{it}+
s_{i}^{X} + \mu_{t}^{X}  + \alpha_{i}+\epsilon_{it}^{X}
\end{equation}
\begin{equation}
d_{it}^{M*} = \gamma_{1}^{M} d_{it-1}^{M} + 
\gamma_{3}^{M} \hat{\omega}_{it}  + \beta_{1}^{M}k_{it}  +\beta_{2}^{L}l_{it}+
s_{i}^{M} + \mu_{t}^{M}  + \alpha_{i}+\epsilon_{it}^{M}
\end{equation}
Here $s_{i}$ is a vector of industry dummies and $\mu_{t}$ is a vector
of time dummies. 

To investigate the presence of lagged complementarity between exporting and
importing, I add another variable to state space: lagged export
choice into the import decision equation and lagged import choice
to export decision equation. Therefore, the Bellman equations that
accounts for lagged complementarity for the export decision is: 
\begin{equation}
V_{it}(S_{it})= \underset{d_{it}^{X}}{\max} \Big(\pi^{D}(S_{it}) + d_{it}^{X}(\pi^{X}(S_{it})) - F(d_{it}^{X},d_{it-1}^{X}, d_{it-1}^{M})
  + \delta \mathbb{E} (V_{it}(S_{it+1})|S_{it}, d_{it}^{X}) \Big)
\end{equation}
where 
 
\begin{equation}
  F(d_{it}^{X},d_{it-1}^{X}, d_{it-1}^{M})=\begin{cases}
   \lambda^{X} * d_{it}^{X}( f^{X} +c^{X}(1-d_{it-1}^{X})), & \text{if $d_{it-1}^{M}= 1$}.\\
   d_{it}^{x}( f^{X} +c^{X}(1-d_{it-1}^{X})) , & \text{if $d_{it-1}^{M}= 0$}.
  \end{cases}
\end{equation}

% $F(d_{it}^{X},d_{it-1}^{X}, d_{it-1}^{M})$ is: 
% \begin{itemize}
% \item $\lambda^{X} * d_{it}^{X}( f^{X} +c^{X}(1-d_{it-1}^{X})$ \hfill for
%   $d_{it-1}^{M}= 1$
% \item $d_{it}^{X}( f^{X} +c^{X}(1-d_{it-1}^{X})$\hfill for
%   $d_{it-1}^{M}= 0$
% \end{itemize}
Here, $\lambda^{X}$ is the degree of cost complementarity that the
lagged decision to import has on the decision to export. If the value
of $\lambda^{X} > 1$, then participating in the import market in
period $t-1$ increases the cost to export in period t. If
$\lambda^{X}=1$, then participating in the import market in
period $t-1$ does not have an effect on the cost to export in period t.
If $0<\lambda^{X}< 1$,     then participating in the import market in
period $t-1$ decreases the cost to export in period t.

Similarly, the Bellman equation for the import decision is: 
\begin{equation}
V_{it}(S_{it})= \underset{d_{it}^{M}}{\max}\big(\pi(S_{it}) - F(d_{it}^{M},d_{it-1}^{M}, d_{it-1}^{X})  + \delta \mathbb{E} (V_{it}(S_{it+1})|S_{it}, d_{it}^{M})\Big)
\end{equation}
where 
\begin{equation}
  F(d_{it}^{M},d_{it-1}^{M}, d_{it-1}^{X})=\begin{cases}
   \lambda^{M} * d_{it}^{M}( f^{M} +c^{M}(1-d_{it-1}^{M})), & \text{if $d_{it-1}^{X}= 1$}.\\
   d_{it}^{M}( f^{M} +c^{M}(1-d_{it-1}^{M})) , & \text{if $d_{it-1}^{X}= 0$}.
  \end{cases}
\end{equation}
% where $F(d_{it}^{M},d_{it-1}^{M}, d_{it-1}^{X})$ is: 
% \begin{itemize}
% \item $\lambda^{M}  d_{it}^{M}( f^{M} +c^{M}(1-d_{it-1}^{M}))$ \hfill for
%   $d_{it-1}^{X}= 1$
% \item $d_{it}^{M}( f^{M} +c^{M}(1-d_{it-1}^{M}))$\hfill for
%   $d_{it-1}^{M}= 0$
% \end{itemize}

Here, $\lambda^{M}$ represents the degree of complementarity on the
cost to import if a firm participated in
the export market in the previous period.  The reduced form
equation of the two Bellman equations that account for lagged
complementarity are the following:\\
\begin{equation}
d_{it}^{X*}=   \gamma_{1}^{X} d_{it-1}^{X} + \gamma_{2}^{M} d_{it-1}^{M}+
\gamma_{3}^{X} \hat{\omega}_{it}  + \beta_{1}^{X}k_{it}  +\beta_{2}^{X}l_{it}+
s_{i}^{X} + \mu_{t}^{X}  + \alpha_{i}+ \epsilon_{it}^{X}
\end{equation}
\begin{equation}
d_{it}^{M*}=   \gamma_{1}^{M} d_{it-1}^{M} + \gamma_{2}^{M} d_{it-1}^{M}+
\gamma_{3}^{M} \hat{\omega}_{it}  + \beta_{1}^{M}k_{it}  +\beta_{2}^{M}l_{it}+
s_{i}^{M} + \mu_{t}^{M}  + \alpha_{i}+ \epsilon_{it}^{M}
\end{equation}
Here it is important to notice that  $\gamma_{2}^{M}$ measures the
cost complementarity effect of lagged importing on
exporting, since the productivity benefit of lagged importing is
accounting for in $\hat{\omega}_{it}$.  Therefore, if $\gamma_{2}^{X}>0$, then this means that
importing in time $t-1$ leads to decrease in cost of exporting at time t.  
In the equation above, if  $\gamma_{2}^{X}>0$, then this means that
exporting in time $t-1$ leads to decrease in cost of importing at time t.
  
I use the dynamic random effects probit specification with Wooldridge
method (\textcite{wooldridge2005simple}) which treats the initial conditions problem by accounting for
the correlation of the initial value with $\alpha$:
$$  \alpha_{i}= \gamma d_{i1}+ \tilde{\alpha}_{i} $$
where $ \tilde{\alpha_{i}} \sim N(0, \sigma_{\alpha}^{2}) $

I use the log value of gross fixed assets as a measure of capital, log
value of wages as a measure of labor and use productivity estimates  of
the Levinsohn-Petrin method in the previous section along with
industry and time dummies.  

\newgeometry{left= 2.5cm, top=4cm}
\begin{center}
\begin{table}[H]
\caption{Dynamic Random Effects Probit (Estimates)}
\label{tab:dynprobit}
\begin{center}
\begin{tabular}{l*{4}{c}}
\hline\hline&\multicolumn{2}{c}{Export
              Decision}&\multicolumn{2}{c}{Import Decision}\\
            &\multicolumn{1}{c}{(1)}&\multicolumn{1}{c}{(2)}&\multicolumn{1}{c}{(3)}&\multicolumn{1}{c}{(4)}\\
            &\multicolumn{1}{c}{Base}&\multicolumn{1}{c}{Lagged}&\multicolumn{1}{c}{Base}&\multicolumn{1}{c}{Lagged}\\
&\multicolumn{1}{c}{}&\multicolumn{1}{c}{Complementarity}&\multicolumn{1}{c}{}&\multicolumn{1}{c}{Complementarity}\\
\hline
        &                     &                     &                     &                     \\
$d_{it-1}^{X}$      &       1.834$^{***}$&       1.786$^{***}$&                     &       0.380$^{***}$\\
            &     (71.18)         &     (68.65)         &                           &     (16.31)         \\
[1em]
$d_{it-1}^{M}$      &                     &       0.370$^{***}$&       1.600$^{***}$&       1.554$^{***}$\\
            &                     &     (14.67)         &     (66.32)         &     (63.85)         \\
[1em]
$\hat{\omega}_{it}$       &       0.216$^{***}$&       0.198$^{***}$&       0.277$^{***}$&       0.266$^{***}$\\
            &     (15.83)         &     (14.70)         &     (21.51)         &     (21.16)         \\
[1em]
$K_{it}$        &      0.0669$^{***}$&      0.0467$^{***}$&       0.109$^{***}$&       0.100$^{***}$\\
            &      (5.17)         &      (3.65)         &      (8.99)         &      (8.51)         \\
[1em]
$L_{it}$     &       0.210$^{***}$&       0.186$^{***}$&       0.213$^{***}$&       0.186$^{***}$\\
            &     (15.09)         &     (13.53)         &     (17.06)         &     (15.20)         \\
[1em]
$d_{i1}^{X}$     &       1.333$^{***}$&       1.264$^{***}$&                     &                     \\
            &     (29.35)         &     (28.62)         &                     &                     \\
[1em]
$d_{i1}^{M}$     &                     &                     &       1.081$^{***}$&       0.986$^{***}$\\
            &                     &                     &     (28.27)         &     (26.62)         \\
[1em]
\_cons      &      -3.264$^{***}$&      -3.122$^{***}$&      -3.655$^{***}$&      -3.617$^{***}$\\
            &    (-25.05)         &    (-24.47)         &    (-28.64)         &    (-29.17)         \\
rho         &       0.392         &       0.373         &       0.348
                                    &       0.321         \\
$\sigma^{2}_{\alpha}$     &       0.804$^{***}$         &       0.771$^{***}$         &       0.731$^{***}$
                                    &       0.687$^{***}$         \\
& (0.0245)& (0.023)& (0.021)& (0.021) \\
Log-Likelihood         &    -14513.0         &    -14406.6         &
                                                                     -15879.7
                                    &    -15749.5         \\
\hline
Industry Dummies & Yes& Yes& Yes& Yes\\
Time Dummies& Yes& Yes& Yes& Yes\\
\hline\hline
\multicolumn{5}{l}{\footnotesize \textit{t} statistics in parentheses}\\
\multicolumn{5}{l}{\footnotesize $^{*}$ \(p<0.05\), $^{**}$ \(p<0.01\), $^{***}$ \(p<0.001\)}\\
\end{tabular}
\end{center}

\end{table}
\end{center}
\restoregeometry
\begin{center}
\begin{table}[H]
\caption{Dynamic Random Effects Probit (Average Marginal Effects)}
\label{tab:dynprobitme}
{
\def\sym#1{\ifmmode^{#1}\else\(^{#1}\)\fi}
\begin{tabular}{l*{4}{c}}
\hline\hline
            &\multicolumn{1}{c}{(1)}&\multicolumn{1}{c}{(2)}&\multicolumn{1}{c}{(3)}&\multicolumn{1}{c}{(4)}\\
            &\multicolumn{1}{c}{} &\multicolumn{1}{c}{} &\multicolumn{1}{c}{} &\multicolumn{1}{c}{} \\
\hline
lagexp      &       0.218\sym{***}&       0.212\sym{***}&                     &      0.0495\sym{***}\\
            &     (53.32)         &     (53.84)         &                     &     (16.34)         \\
[1em]
lpres       &      0.0257\sym{***}&      0.0235\sym{***}&      0.0362\sym{***}&      0.0347\sym{***}\\
            &     (15.99)         &     (14.82)         &     (21.92)         &     (21.55)         \\
[1em]
lgfa        &     0.00797\sym{***}&     0.00553\sym{***}&      0.0142\sym{***}&      0.0130\sym{***}\\
            &      (5.19)         &      (3.66)         &      (9.03)         &      (8.54)         \\
[1em]
lsalary     &      0.0250\sym{***}&      0.0220\sym{***}&      0.0279\sym{***}&      0.0243\sym{***}\\
            &     (15.20)         &     (13.62)         &     (17.27)         &     (15.37)         \\
[1em]
lagimp      &                     &      0.0439\sym{***}&       0.209\sym{***}&       0.202\sym{***}\\
            &                     &     (14.65)         &     (53.75)         &     (53.94)         \\
\hline
\(N\)       &       62485         &       62485         &       62485         &       62485         \\
\hline\hline
\multicolumn{5}{l}{\footnotesize \textit{t} statistics in parentheses}\\
\multicolumn{5}{l}{\footnotesize \sym{*} \(p<0.05\), \sym{**} \(p<0.01\), \sym{***} \(p<0.001\)}\\
\end{tabular}
}

\end{table}
\end{center}

Table \ref{tab:dynprobit} and \ref{tab:dynprobitme} displays the
estimates and average marginal effects of the dynamic random
effects probit model respectively. 

Columns 1 and 3 display the result for the base
reduced model of importing and exporting  whereas columns 2 and 4
display the estimates when accounting for the lagged complementarity
between exporting and importing.

The likelihood ratio test  rejects the base model in favor of the
model that accounts for the lagged complementarity in the case of 
the export decision as well as the import decision. Therefore, I
interpret the model which accounts for the lagged complementarity.
 
% \begin{enumerate}
% \item Export: \\$H_{o}$: Base Model, $H_{1}$:Model with lagged discrete
%   import variable.
% $LR= 2[ln(model2) - ln(model1)] = 2[-14406.6 + +14513.0] = 106.4$
% The critical value of $\chi^{2}_{1;0.95}= 3.84 $. Therefore $H_{o}$ is
% rejected. 
% \item Import:\\ $H_{o}$: Base Model, $H_{1}$:Model with lagged discrete
%   export variable. 
% $LR= 2[ln(model2) - ln(model1)] = 2[-15749.5 + 15879.7 ] = 130.19$
% The critical value of $\chi^{2}_{1;0.95}= 3.84 $. Therefore $H_{o}$ is
% rejected. 
% \end{enumerate}
% In both cases, the model that accounts for the lagged complementarity
% performs better than the base model. 

    The state dependence parameter for the decision to export is positive
  significant at 1\% level with the highest magnitude amongst all of
  the coefficients. This suggests that there is persistence in
  exporting behavior which confirms that sunk-cost hypothesis. The average marginal effect of the lagged
  decision to export is the higest at 0.218. This  means that there is 20\% increase
  in probability to export if a firm has exported in the previos
  period.  Productivity also has a significant and
  positive effect on exporting which provides evidence of
  self-selection of high productivity into exporting. Capital and Labor
  also have a positive and significant effect, which tells us that a
  bigger firm has a higher chance of participating in the export
  market. The lagged import coefficient in column 2 is
  significant  at 1\% level and positive which suggests  that importing
  at time t increases the probability of exporting at time $t+1$ and
   the presence of cost complementarity. The coefficient
estimates of the initial export is high and highly significant,
suggesting that it has correlation with the decision to export. The value of
  $\sigma^2_{\alpha}$ in the table does not account for the
  contribution of the intial export decision. Therefore, the estimated
  variance of 
  unobserved heterogeneity $\sigma^2_{\alpha}= \lambda^2 *
  \hat{Var(d_{i1}}^{X})+\hat{\sigma^2}_{\tilde{\alpha}}= 1.16$. This
  means that individual effects capture about 54\% of the unexplained
  variance. This suggests there are variables other than the ones used
  in the estimation that contribute towards the export market
  participation decision. 


 The results for the import decision are quite similar to the
 results for the export decision. The state dependence parameter is the most highest
  amongst all variables and
  significant at 1\% level. This suggests that there is  persistence in
  importing behavior and confirms the sunk cost hypothesis for import decision. The average marginal effect of the lagged
  decision to export is 0.202 and has the highest effect amongst all
  of the variables.  Productivity is also significant at 1\%
  and has a 
  positive effect on importing which provides evidence in favor of
  self-selection of high productivity into importing. Capital and Labor
  also have a positive and significant effect, which provides evidence
  in favor of 
  bigger firms having a higher probability of import participation.  The lagged export coefficient is
  significant and positive and this provides evidence  that exporting
  at time t increases the probability of importing at time $t+1$. It
  has the second highest value of the average marginal effects.  This
  is evidence in favor of 
  lagged cost complementarity hypothesis for
  import decision.  The initial import variable is high and
  significant in this case.  The
  unobserved heterogeneity is high as it explains about 48\% of the unexplained
  variation in the data. This value is lower than the value
  when the lagged export decision is not included in the dynamic
  specification (Model 1) 

Table \ref{tab:dynprobitacf} in Appendix \ref{apendix:random}displays the estimation results for the same
equation with ACF productivity estimates. The coefficients in this
table are quite similar to the results above. 

These results provide evidence in favor of  the sunk-cost hypothesis,
self-selection of higher productivity and bigger firms into
exporting and importing as well as lagged cost
complementarity.
 
However, this estimation equation cannot estimate if there is  contemporaneous cost complementarity between exporting
and importing. The next section shows the results of a dynamic
bivariate probit model which accounts for contemporaneous cost complementarity
as well. 

% I use the following equations  to verify that more productive firms
% self-select into participating in the export/import market:
% \begin{equation}
% \hat{log(TFP)_{t-j}} = \gamma_{1}log(export)_{it}+ \gamma_{2}log(import)_{it} +
%  \beta c_{i,t-j}
% \end{equation}

% \begin{equation}
% \hat{log(TFP)_{t-j}} = \gamma_{1}d_{it}^{X}+ \gamma_{2}d_{it}^{M} +
% \gamma_{3}d_{it}^{X}d_{it}^{M} + \beta c_{i,t-j}
% \end{equation}
% where $c_{i,t-j}$ contains log of capital and labour. I estimate the
% equation mentioned above for three time periods $j=1,2 \& 3$ and for
% the discrete decision as well as the value of exports/imports.  The
% coefficients are estimated using fixed-effects regression. 
% Table \ref{discprod} and Table \ref{contprod} display  $\gamma$ estimates for equation 2 and 3
% respectively
  
% \input{./TABLES/discprod.gen}
% \input{./TABLES/contprod.gen}
% In the discrete case, productivity of firms which export in year t is
% 12.5 \%, 7\% and 4.2 \% higher than non-exporting firms in in year
% t-1,t-2 and t-3 respectively. And the productivity of firms which
% import in year t is 13.8 \%, 8\% and 4.9 \% higher than non-nonimporting firms in in year
% t-1,t-2 and t-3 respectively. The interaction variables are
% not significant in all the three cases. This suggests that for firms
% to participate in both the markets, their lagged productivity needs to
% be higher than firm who participate in either the export or the export
% market. Another interesting feature is that firms that only import in
% year t have higher lagged productivity than firms that only export in
% year t.

% Tables\ref{discprod} and  \ref{contprod} provide evidence that lagged productivity at for all the three
% time periods before is higher when the firm participates in the export
% market in year t. This gives evidence of self-selection of firms into
% exporting and importing as the lagged productivity for three
% consecutive time periods before exporting/importing has a
% significantly positive value. 
\subsection{Complementarity between exporting and importing(Lagged and
Contemporaneous)}\label{sec:biprobit}

The Bellman equation of a profit maximising firm deciding to export
and import simultaneously  is the following:  

\begin{equation}
V_{it}(S_{it}) = \underset{d_{it}}{\max} \big(\pi_{it}^{d}(S_{it}) +d_{it}^{X}*(\pi_{it}^{X}(S_{it})) +
F(d_{it}, d_{it-1}) + \beta \mathbb{E}(V_{it}(s_{it+1})|s_{it}, d_{it}) \big)
\end{equation}
where $d_{it}= (d_{it}^X, d_{it}^M)$ and $S_{it}= (k_{it}, l_{it},
\omega_{it}, d_{it-1}^X, d_{it-1}^M)$.  $F(d_{it}, d_{it-1})$ is the
fixed/sunk costs the firms must pay to export and import and it takes the
following form:

$ F(d_{it}^{X},d_{it}^{M},d_{it-1}^{X}, d_{it-1}^{X})=$
\begin{equation}
  F(d_{it}^{M},d_{it-1}^{X}, d_{it-1}^{X})=\begin{cases}
   \lambda^{M} * d_{it}^{M}( f^{M} +c^{M}(1-d_{it-1}^{M})), & \text{if $d_{it-1}^{X}= 1$}.\\
   d_{it}^{M}( f^{M} +c^{M}(1-d_{it-1}^{M})) , & \text{if $d_{it-1}^{X}= 0$}.
  \end{cases}
\end{equation}
\begin{equation}
  F(d_{it}^{X},d_{it-1}^{X}, d_{it-1}^{M})=\begin{cases}
   \lambda^{X} * d_{it}^{X}( f^{X} +c^{X}(1-d_{it-1}^{X})), & \text{if $d_{it}^{M}= 1$}.\\
   d_{it}^{X}( f^{X} +c^{X}(1-d_{it-1}^{X})) , & \text{if $d_{it-1}^{M}= 0$}.
  \end{cases}
\end{equation}
\begin{equation}
  F(d_{it}^{X},d_{it}^{M},d_{it-1}^{X}, d_{it-1}^{X})=\begin{cases}
  \lambda^{B}[f^{X} + f^{M} + c^{X}(1 - d_{it-1}^X) + c^{M}(1 -
  d_{it-1}^M)] , & \text{if $(d_{it}^{X},d_{it}^{m})= (1,1)$}.\\
   0, & \text{if $(d_{it}^{X},d_{it}^{m})= (1,1)$}.
  \end{cases}
\end{equation}

% $F(d_{it}, d_{it-1})= $
% \begin{enumerate}
% \item $\lambda^{M}  d_{it}^{M}( f^{M} +c^{M}(1-d_{it-1}^{M}))$ \hfill for
%   $(d_{it}^X, d_{it}^M,d_{it-1}^{X},d_{it-1}^M)= (0,d_{it}^M,1,d_{it-1}^M)$
% \item $d_{it}^{M}( f^{M} +c^{M}(1-d_{it-1}^{M}))$\hfill for
%   $(d_{it}^X, d_{it}^M,d_{it-1}^{X},,d_{it-1}^M)= (0,d_{it}^M,0,d_{it-1}^M)$
% \item $\lambda^{X}  d_{it}^{X}( f^{X} +c^{M}(1-d_{it-1}^{X}))$ \hfill for
%   $(d_{it}^X, d_{it}^M,d_{it-1}^X,d_{it-1}^{M})= (d_{it}^X,0,d_{it-1}^X,1)$
% \item $d_{it}^{X}( f^{X} +c^{X}(1-d_{it-1}^{X}))$\hfill for
%   $(d_{it}^X, d_{it}^M,d_{it-1}^X,d_{it-1}^{M})= (d_{it}^X,0,d_{it-1}^X,0)$
% \item   $\lambda^{B}[f^{X} + f^{M} + c^{X}(1 - d_{it-1}^X) + c^{M}(1 -
%   d_{it-1}^M)]$   for $ (d_{it}^X, d_{it}^M,d_{it-1}^X,d_{it-1}^{M}) =
%   (1,1, d_{it-1}^X,d_{it-1}^M) $
% \end{enumerate}
Here $f^{X}$ is the fixed cost of exporting,$C^{X}$ is the sunk cost
of exporting, $f^{M}$ is the fixed cost of importing and $f^{M}$ is the
fixed cost of importing. $\lambda^{X}$ captures the degree of cost 
complementarity that lagged decision to  import has on decision to
export, $\lambda^{M}$ captures the degree of cost 
complementarity that lagged decision to export has on decision to
import and  $\lambda^B$ captures the degrees of
contemporaneous complementarity between exporting and importing. 

The reduced form model of the bellman equation is a dynamic bivariate
probit model.  Then, the bivariate dynamic probit model  takes the following form:

\begin{equation}
  d_{it}^{X}=\begin{cases}
   1 , & \text{if $d_{it}^{X*}\geq 0$}.\\
   0 , & \text{if $d_{it}^{X*}<  0$}.
  \end{cases}
\end{equation}

\begin{equation}
  d_{it}^{M}=\begin{cases}
   1 , & \text{if $d_{it}^{M*} \geq  0$}.\\
   0 , & \text{if $d_{it}^{M*}<  0$}.
  \end{cases}
\end{equation}
The discrete decision of exporting and importing is modelled as a function of previous import and
export status controlling for lagged firm characteristics and industry and time fixed
effects. 
\begin{equation}
d_{it}^{X*} = \gamma_{1}^{X} d_{it-1}^{X} + \gamma_{2}^{X} d_{it-1}^{M}+
\gamma_{3}^{X} \hat{\omega}_{it}  + \beta_{1}^{X}k_{it}  +
s_{i}^{X} + \mu_{t}^{X}  + \epsilon_{it}^{X}
\end{equation}
Here $\gamma_{1}$ identifies the state dependence coefficient, $\gamma_{2}$ accounts for
the fact that participating in one activity in time t-1 improves the
odds of participating in the other at time t (lagged complementarity),$\gamma_{3}$ accounts for
the self-selection mechanism, $\beta_{1}$ accounts for capital at time
t-1 and $s_{i}^{M}$  $\mu_{t}^{M}$ are industrial
and time dummies respectively.

The reduced form specification for import is similar to the export
equation:
\begin{equation}
d_{it}^{M*} = \gamma_{1}^{M} d_{it-1}^{M} + \gamma_{2}^{M} d_{it-1}^{X}+
\gamma_{3}^{M} \hat{\omega}_{it}  + \beta_{1}^{M}l_{it}  +
s_{i}^{M} + \mu_{t}^{M}  + \epsilon_{it}^{M}
\end{equation}

The bivariate specification also allows for the
contemporaneous complementarity between the two choices as
$\epsilon_{it}^{X}$ and $\epsilon_{it}^{M}$ are allowed to be
correlated. This gives gives the following form to the error
structure: 


\[\begin{pmatrix}
\epsilon_{it}^{X} \\
\epsilon_{it}^{M}
\end{pmatrix}\sim N\left(\begin{pmatrix}
0 \\
0
\end{pmatrix},\begin{pmatrix}
1 & \rho \\
\rho & 1
\end{pmatrix}\right)
\]
The estimated $\rho$ measures the correlation between the unobserved
errors between the two activities. Therefore, this provides evidence
of contemporaneous complementarity if it significantly positive after
controlling for different effects. 
This model specification has been used to test the contemporaneous relationship
by \textcite{aristei2013firms}, \textcite{aw2007export} and \textcite{manez2015dynamic}. 

This model has a few drawbacks: it does not account for individual
heterogeneity ($\alpha_{i}$) and it does not account for the  initial
conditions problems. Therefore, the model assumes that $d_{i1}$ is 
endogenously given and that the firm characteristics and industry and time  dummy variables account
for the individual heterogeneity between firms.

Table \ref{tab:biprobit} displays the results for dynamic probit specification. All
  the coefficients except the capital coefficient is significant at
  1\% level and have the same sign 
  to the coefficients in the previous section. However, all of the coefficients
  in import decision are significant. The
  state-dependence coefficient has the strongest effect amongst all
  the variables, suggesting that there is persistence in both the
  activities and high sunk cost. There is a positive effect of 
  productivity on both activities, providing further evidence of
  self-selection of firms into exporting and importing. The
  coefficients of labor and productivity are positive and
  quite similar to previous section. Firms which were importing in the previous year are more
  likely to be exporters and firms which were exporting in the
  previous year are more likely to be importing this year.

 The likelihood-ratio test with the null hypothesis that
  the correlation between the unobserved errors is 0 is rejected at
  1\% significance level. This means that there is a significant
  increase in the log likelihood of the model in a bivariate probit
  specification as compared to two independent probit models.
  Therefore, the estimated value of $\rho$ is 0.438 and
  is significantly different than zero. This suggests that there is
  contemporaneous cost complementarity between exporting and importing 
as shocks that lead a firm to participate in one activity tend
to lead it to participate in both.


\begin{center}
\begin{table}[H]
\caption{Dynamic Bivariate Probit (Estimates)}
\label{tab:biprobit}
\begin{center}
\begin{tabular}{l*{2}{c}}
\hline\hline
            &\multicolumn{1}{c}{(1)}&\multicolumn{1}{c}{(2)}\\
            &\multicolumn{1}{c}{Export
              Decsion}&\multicolumn{1}{c}{Import Decision}\\
\hline\\

$d_{it-1}^{X}$  &          2.539$^{***}$    &   0.397$^{***}$ \\
            &    (1    (0.0312)             &(0.0273)         \\
[1em]                                                        
$d_{it-1}^{M}$      &      0.360$^{***}$    &   2.175$^{***}$\\
            &          (0.0278)             &(0.0335)         \\
[1em]                                                        
$\hat{\omega}_{it}$  &     0.103$^{***}$     &  0.165$^{***}$\\
            &          (0.0152)             &(0.0150)         \\
[1em]                                                        
$k_{it}$       &        0.00873              & 0.0788$^{***}$\\
            &          (0.0154)             &(0.0122)         \\
[1em]                                                        
$l_{it}$     &            0.135$^{***}$     &  0.134$^{***}$\\
            &          (0.0182)             &(0.0173)         \\
[1em]                                                        
Constant      &          -2.212$^{***}$     & -2.768$^{***}$\\
            &           (0.111)             &(0.0809)         \\
\hline
$corr(\epsilon_{it}^{X},\epsilon_{it}^{M}) $      &       0.439$^{***}$\\
            &    (0.0173)         \\
LR test& $corr(\epsilon_{it}^{X},\epsilon_{it}^{M})$, $\chi^{2}(1)= 641.056$&\\
& p=0.000&\\
Industry Dummies & Yes& \\
Time Dummies& Yes& \\
\hline\hline
\textit{Note:}&\multicolumn{2}{r}{\footnotesize  Robbust standard errors in parentheses}\\
&\multicolumn{2}{r}{\footnotesize $^{*}$ \(p<0.05\), $^{**}$ \(p<0.01\), $^{***}$ \(p<0.001\)}\\
\end{tabular}
\end{center}

\end{table}
\end{center} 

Table \ref{tab:biprobitacf} in Appendix \ref{apendix:bivariate} displays the results for the ACF
productivity estimates and have  similar result to the table
above. 
 
\section{Conclusion}\label{sec:conclusion}
In this paper, I analyse the dynamic linkages between exports, imports
and productivity.  

First, I estimate the the effect of the decision to export and import
on productivity. This is done by endogenously accommodating the the
decision to export and import in the productivity evolution process in
the methods suggested by \textcite{levinsohn2003estimating} and
\textcite{ackerberg2006structural}. I find that the decision to export
does not have a significant effect on productivity whereas the
decision to import has a significant and positive effect  on
productivity (3.6\% and 2.7\% according to LP and ACF method respectively) 

Second, I estimate a dynamic random effects probit model on the
decision to export and import. I use lagged decision to import and
export, capital, labor and  productivity estimates as covariates. It
is found that there is high levels persistence of export and import
and capital, labor and  productivity have a positive and significant
impact on the decision to export and import. Moreover, the lagged
decision to export has a positive effect on the probability to import
and vice versa. 

Third, I estimate a dynamic bivariate probit model  on the
decision to export and import and use the same covariates as in the
dynamic random effects probit estimation procedure. The covariates
have the same sign and therefore the same effect as in the dynamic
random effects probit estimation procedure.
The interesting result from this estimation is that both decisions are
correlated as the unobserved errors of the two decisions are
signficantly correlated with a value of 0.439.  



% The results from this section and descriptive statistics suggest a few
% overall themes of the data: 
% \begin{itemize}
% \item Learning-by-doing: I estimated  productivity using
%   LP (\textcite{levinsohn2003estimating}) and ACF
%   (\textcite{ackerberg2006structural}) such 
% that productivity  is
%   endogenously  dependent on the
%   lagged export and import choices.  I get the following results for
%   the the two trading activities:
The results lead to following conclusions for exporting behavior:
i): Firms do not display learning-by-exporting,
ii): Persistence in the exporting behavior  and 
  self-selection of higher productivity firms into
   exporting  
iii):  Lagged export decision has a positive decision on importing.
And for the importing behavior:
i): Firms  display learning-by-importing,
ii): Persistence in the importing behavior and self-selection of higher productivity firms into
   exporting and
iii):  Lagged export decision has a positive decision on
importing.
Moreover, there is contemporaneous cost complementarity between the
decision to export and import. 

\section{Further Work}

The dynamic bivariate probit model in this paper does not account for
individual heterogeneity and the initial conditions
problem. Therefore, estimating a dynamic bivariate model that
accommodates individual heterogeneity by using random effects and the
endogeneity of initial conditions problem with the help of
\textcite{wooldridge2005simple} will help provide more robust
results for contemporaneous cost complementarity between exporting and
importing. 

Future work in this field would be to estimate the fixed and sunk
costs of the trading activities as well as their cost
complementarity with the help of a dynamic model in the nature of
\textcite{kasahara2013productivity} and \textcite{aw2011}. And then
run a counter factual simulation to see the effect of an increase in the
costs to exporting on importing and vice-versa.  

\newpage
% \item Export:  as the
%   coefficient of discrete choice of lagged export decision does not
%   have a significant effect on productivity. 
% \item Import: Firms display learning-by-importing as the
%   coefficient of discrete choice of lagged import decision does
%   have a significantly positive effect on productivity. 
% \item Both: Firms do not display any extra productivity benefit of participating in both
%   the export/import market other than the productivity benefit of
%   importing. 
% \end{enumerate}
% \item Sunk cost and Self-Selection: I estimated a dynamic bivariate
%   random effects models that accounted for the initial conditions
%   problems using \textcite{wooldridge2005simple}, to get the following results:
% \begin{enumerate}
% \item Sunk Cost: The high coefficient of the lagged variable suggests
%   that there is persistence in the exporting/importing, which provides
%   evidence of the presence of sunk cost in importing and exporting. 
% \item Self-Selection:  The coefficient of productivity is
%   significantly positive which provides evidence of the presence of
%   self-selection of higher productivity firms into
%   exporting/importing. It is also found that bigger firms i.e firms
%   with more capital and labor have a higher probability to
%   export/import. 
% \item Lagged export decision has a positive decision on importing and
%   vice-versa. This confirms the presence of lagged complementarity. 
% \item Individual heterogeneity plays a vital role in deciding to
%   export/import.
% \end{enumerate} 
% \item Complementarity between exporting and importing: I estimated a
%   bivariate dynamic probit regression of discrete choice of
%   exporting/importing on their lagged values, firm characteristics and
%   industry and time dummies to get the following results:
% \begin{enumerate}
% \item There is strong correlation between the unobserved errors 
%   indicating the presence of contemporaneous complementarity. 
% \item The other results are quite similar to the dynamic probit random
%   effects model. 
% \end{enumerate}
% \end{itemize}

% \section{Model}

% I use a model inspired from  \textcite{aw2011} and \textcite{kasahara2013productivity}. 

% \subsection{Static Decision}

% A firm i has a standard Cobb-Douglas Production Function and faces a
% marginal cost:

% \begin{equation}
% ln c_{it} = \beta_{o} + \beta_{k}ln k_{it} + \beta_{w}ln w_{t} + \omega_{it}
% \end{equation}
% where 
% \begin{itemize}
% \item $K_{it}$ is the unit of output
% \item $w_{t}$ is a vector of variable input prices common to all firms
% \item $\omega_{it}$ is the productivity shock
% \end{itemize}

% A firm faces a constant elasticity of demand (CES) function assumed to
% be of the Dixit-Stiglitz form :

% \begin{equation}
% Q_{it}^{D} = Q_{t}^{D}(\frac{P_{it}^{D}}{P_{t}^{D}})^{\eta_{d}}= \Phi_{t}^{D} (p_{it}^{D})^{\eta_{d}}
% \end{equation} 
% where $Q_{t}^{d}$ is the industry aggregate output, $P_{t}^{d}$ is
% the price index and $P_{it}^{d}$ is the firm i's price, $\eta_{D}$ is
% the constant elasticity of demand. So, the firms demand depends upon
% aggregate market conditions $\Phi_{t}^{D}$

% The demand function in the export market has a similar structure
% except that it also depends on an industry-specific demand shifter: 
% \begin{equation}
% Q_{it}^{X} =
% Q_{t}^{X}(\frac{P_{it}^{X}}{P_{t}^{X}})^{\eta_{d}}exp(z_{it})= \Phi_{t}^{X} (p_{it}^{X})^{\eta_{d}}exp(z_{it})
% \end{equation} 
% where $z_{it}$ is the unobserved firm specific demand
% shock. 

% Equation 2 can be used to obtain an expression for $P_{it}$ and a
% firms domestic revenue is $R_{it} = P_{it}Q_{it}$, and inserting price
% into the revenue function and taking a log to get the revenue function
% in the domestic market:

% \begin{equation}
% ln r_{it} = (\eta_{d} +1) ln \frac{\eta_{d}}{\eta_{d} +1}  + ln
% \Phi_{t}^{D} + (\eta_{d} +1)(\beta_{k}K_{it} + \beta_{w} ln w_{t} +
% \omega_{it}) 
% \end{equation}
%  The revenue function for the export market can be similarly derived
%  to get:
% \begin{equation}
% ln r_{it} = (\eta_{d} +1) ln \frac{\eta_{d}}{\eta_{d} +1}  + ln
% \Phi_{t}^{X} + (\eta_{d} +1)(\beta_{k}K_{it} + \beta_{w} ln w_{t} +
% \omega_{it})  +  z_{it}
% \end{equation}


% \textcite{das2007market} display a relation between profits and revenue. I
% use this estimate the constant demand of elasticity in both the
% domestic and export market. 
% In the domestic market, the profits are: 
% \begin{equation}
% \pi_{it}^d = \frac{1}{\eta_{d}} r_{it}^{d}(K_{it}, \omega_{it}, \Phi_{t}^{D})
% \end{equation} 

% In the export market, the profits are: 
% \begin{equation}
% \pi_{it}^X = \frac{1}{\eta_{X}} r_{it}^{X}(K_{it}, \omega_{it}, \Phi_{t}^{X})
% \end{equation} 

% \begin{equation}
% tvc_{it} = r_{it}^{D}(1 + \frac{1}{\eta_{D}}) + r_{it}^{D}(1 +
% \frac{1}{\eta_{D}}) + \epsilon_{it}
% \end{equation}


% \subsection{Transition of Productivity}

% The firm-level productivity is allowed to the be endogenously affected
% by the firms decision to export and import just as before. . Therefore, the law of
% motion of productivity is:

% \begin{equation}
% \omega_{it} = g(\omega_{it-1}, d_{it-1}^{X}, d_{it-1}^{M}) + \nu_{it}
% \end{equation}

% \begin{equation}
% \omega_{it} = \alpha_{o} + \alpha_{1}\omega_{it-1} +
% \alpha_{2}\omega_{it-1} + \alpha_{3}\omega_{it-1}^{2}+
% \alpha_{4}d_{it-1}^{X} + \alpha_{5} d_{it-1}^{M} + \alpha_{6}d_{it-1}^{X}d_{it-1}^{M}  \nu_{it}
% \end{equation}

% where $d_{it-1}^{X}$ is an indicator function of the firms lagged export
% participation, $d_{it-1}^{M}$ is an indicator function of the firms lagged import
% participation and $\nu_{it}$ is an iid shock to the productivity. The
% lagged export and import indicator variables allow for
% learning-by-exporting and productivity benefits from importing. 

% The model assumes that productivity is only affected by the intensity
% of export/importing but is only dependent on the decision. 

% The firm-specific demand shock$z_{it}$, and industry index $\Phi_{t}^{X}$  and
% $\Phi_{t}^{X}$ is modelled as an exogenous AR(1) process. 


% \subsection{Dynamic Model}
 
% Firm must pay a fixed/sunk costs of trading. Let $d_{i,t}^X$ be the
% indicator function of participation in the export market and
% $d_{i,t}^M$ be the indicator function of participation in the import
% market. Then the total costs (sunk and fixed) paid by firm i in period t is given by:


% $F(d_{it}, d_{it-1})= $
% \begin{enumerate}
% \item   $f^{x} + c^{X}(1 - d_{it-1}^X)$\hfill  for $ (d_{it}^X, d_{it}^M) =
%   (1,0) $
% \item   $f^{M} + c^{M}(1 - d_{it-1}^M)$\hfill  for $ (d_{it}^X, d_{it}^M) =
%   (0,1) $
% \item   $\lambda[f^{x} + f^{M} + c^{X}(1 - d_{it-1}^X) + c^{M}(1 -
%   d_{it-1}^M)]$  \hfill for $ (d_{it}^X, d_{it}^M) =
%   (1,1) $
% \item   0  \hfill for $ (d_{it}^X, d_{it}^M) =
%   (0,0) $
% \end{enumerate}
% Here $f^{X}$ is the fixed cost of exporting,$C^{X}$ is the sunk cost
% of exporting, $f^{M}$ is the fixed cost of importing, $f^{M}$ is the
% fixed cost of importing and $\lambda$ captures the degrees of
% complementarity between exporting and importing. 

% $$ S_{it} = (\omega_{it}, K_{it}, d_{it-1}^{X}, d_{it-1}^{M},
% \Phi_{t}^{X}, \Phi_{t}^{D}, z_{it})$$

% \begin{equation}
% V_{it}(S_{it}) = max_d(\pi_{it}^{d} +d_{it}\pi_{it}^{X} + F(d_{it}, d_{it-1}) + \beta E(V_{it}(s_{it+1}|s_{it})))
% \end{equation}

% Therefore, for any state vector, the marginal benefit of exporting is
% equal to:

% \begin{equation}
% MBE(S_{it}|d_{it-1}) = \pi_{it}^X + V_{it}(s_{it}|e_{it-1}=1) - V_{it}(s_{it}|e_{it-1}=0)
% \end{equation}

% Therefore, for any state vector, the marginal benefit of importing is
% equal to:

% \begin{equation}
% MBM(S_{it}|d_{it-1}) =  V_{it}(s_{it}|M_{it-1}=1) - V_{it}(s_{it}|M_{it-1}=0)
% \end{equation}

% \begin{equation}
%   V_{it}(s_{it}) = \int (\pi_{it}^D + max_{e_{it}}\{( \pi_{it}^D +
%   e_{it-1}\gamma_{it}^{F} - (1- e_{it})\gamma_{it}^{S}) + V_{it}^{E}(s_{it}) ,
%   V_{it}^{D}(s_{it})\}) dG^{\gamma}
% \end{equation}

 
% \begin{equation}
%   V_{it}(s_{it}) = \int (\pi_{it}^D + max_{e_{it}}\{( \pi_{it}^D +
%   e_{it-1}\gamma_{it}^{F} - (1- e_{it})\gamma_{it}^{S}) + V_{it}^{E}(s_{it}) ,
%   V_{it}^{D}(s_{it})\}) dG^{\gamma}
% \end{equation}


% \begin{equation}
%   V_{it}^E(s_{it}) = \int ( max_{m_{it}}\{ \beta E_{t}
%   V_{it+1}(s_{it+1}|e_{it}=1, m_{it} =1) - m_{it-1} \gamma_{it}^{mf} , \beta E_{t}
%   V_{it+1}(s_{it+1}|e_{it=1}=1, m_{it} =0) \} dG^{\gamma}
% \end{equation}


% \begin{equation}
%   V_{it}^D(s_{it}) = \int ( max_{m_{it}}\{ \beta E_{t}
%   V_{it+1}(s_{it+1}|e_{it}=0, m_{it} =1) - m_{it-1} \gamma_{it}^{mf} , \beta E_{t}
%   V_{it+1}(s_{it+1}|e_{it}=0, m_{it} =0) \} dG^{\gamma}
% \end{equation}

\printbibliography[omitnumbers=false]
\newgeometry{top=3cm}
\section{Appendix}

\subsection{Appendix (Descriptive Statistics)}\label{sec:fuel}
\begin{center}
\begin{table}[H]
\begin{center}
\caption{Summary Statistics of Expenditure on Raw Material (log)}
\label{tab:lrawmat}
\begin{tabular}{c}
 \includegraphics{./PICS/denslrawmat.pdf}   \\ 
 \input{./TABLES/sumstatslrawmat.gen}  \\  
\end{tabular}
\end{center}
\end{table}
\end{center}
\begin{center}
\begin{table}[H]
\begin{center}
\caption{Summary Statistics of Expenditure on Power and Fuel (log)}
\label{tab:lfuel}
\begin{tabular}{c}
 \includegraphics{./PICS/denslpower.pdf}   \\ 
 \input{./TABLES/sumstatslpower.gen}  \\  
\end{tabular}
\end{center}
\end{table}
\end{center}
\restoregeometry

\subsection{Appendix (Productivity Estimation OP)}\label{prodest}
\label{op}

 \textcite{olley1992dynamics} (OP) use the
following strategy to estimate the Cobb-Douglas function: 


The log transformation of the production function is written as : 
\begin{equation}
y_{it} = \beta_{o} + \beta_{l}l_{it} + \beta_{k}k_{it} + \omega_{it} + \epsilon_{it} 
\end{equation}
where $l_{it}$ is labour, $k_{it}$ is capital, $m_{it}$  is
the demand for materials, $\omega_{it}$ is firm-level-productivity
observable to the firm and $\epsilon_{it}$ are shocks to production.\\
The optimal investment decision of the firm is characterised by the following
function:
\begin{equation}
i_{it}= f_{k}(l_{it-1}, k_{it}, \omega_{it})
\end{equation}
The optimal labor decision is characterised by the following function:
\begin{equation}
l_{it}= f_{k}(l_{it-1}, k_{it}, \omega_{it})
\end{equation}

These are the assumptions made by \textbf{OP}:
\begin{enumerate}
\item $f_{k}(l_{it-1}, k_{it}, \omega_{it})$ is invertible in
  $\omega_{it}$
\item $\omega_{it}$ follows a first order markov process 
\item Investment at period i is not active until period $t+1$:
  $k_{it+1}= (1-\delta)k_{it} + i_{it}$
\item Labor is a perfectly flexible input i.e there are no labor
  adjustment cost
\end{enumerate}

The estimation procedure proceeds in two step:
\begin{enumerate}
\item Estimation of $\beta_{k}$:
Since $\omega_{it}=f_{k}^{-1}(l_{it-1},k_{it},i_{it})$ and inserting
this in the Cobb-Douglas equation to get: 
$$ y_{it} = \beta_{l}l_{it} + \phi_{it}(l_{it-1},i_{it},k_{it})$$
where $\phi_{it}(l_{it-1},i_{it},k_{it}) =  \beta_{k}k_{it}+ f_{k}^{-1}(l_{it-1},k_{it},i_{it})$
$\phi_{it}(l_{it-1},i_{it},k_{it})$ is approximated using a polynomial
expression and $beta_{l}$ is estimated by using OLS on the above
equation
\item Estimation of $\beta_{k}$:
Since $\omega_{it}$ follows a first order markov process, it can be
written as: 
$$ \omega_{it} = h(\omega_{it-1}) + e_{it}$$
Then $\phi_{it}$ can be written as: 
$$ \phi_{it} = \beta_{k}k_{it} + h(\omega_{it-1}) + e_{it}$$
$$ \phi_{it} = \beta_{k}k_{it} + h(\phi_{it-1}- \beta_{k}k_{it-1}) + e_{it}$$
The unknown form of function h is approximated by quadratic function
and for any value of $\beta_k$ to get:
$$ \hat{\phi_{it}} = \beta_{k}k_{it} +\gamma_{0}
\gamma_{1}\hat{\omega_{it-1}^{\beta_{k}}}+
\gamma_{2}(\hat{\omega_{it-1}^{\beta_{k}}})^{2}
+ \gamma_{3}(\hat{\omega_{it-1}^{\beta_{k}}})^{3} $$
 This expression is minimised to get the estimate of $\beta_{k}$. 
\end{enumerate}
% \subsubsection{\textcite{levinsohn2003estimating} (LP)}

% \subsubsection{\textcite{ackerberg2006structural}(ACF)}
\newpage
\newgeometry{top=3cm}
\subsection{Appendix (Learning-by-doing)}\label{apendix:lbd}

\input{./TABLES/regLPcont.gen}
\input{./TABLES/prodcont.gen} 
\restoregeometry
\input{./TABLES/regACFcont.gen}
\input{./TABLES/prodACFcont.gen} 


\subsection{Appendix (Dynamic Random Effects Probit  Model)}\label{apendix:random}
Let i be the unit and t be the time. A dynamic random effects probit
is written as: 
$$ y_{it}^{*} = \gamma y_{i,t-1} + x_{it}'\beta + \alpha_{i} +
\epsilon_{it}; y_{it}=1[y_{it}^{*} > 0]$$
where $\gamma$ is the state dependence parameter.



There are three ways to estimate the above equation: 
\begin{enumerate}
\item Treat $y_{i1}$ as exogenously given and do not explain it
\item Heckman Method
\item Wooldridge Method
\end{enumerate} 
I use the Wooldridge method which assumes that $\alpha_{i}$:
$$\alpha_{i} = \lambda y_{i1} + \tilde{\alpha_{i}}$$

Assumptions: 

\begin{enumerate}
\item The i-units are a random sample
\item $\epsilon_{it} \sim N(0,1)$ is independent of $x_{i}$
\item $\tilde{\alpha_{i}} \sim N(0,\sigma_{alpha}^{2})$ is independent of
  $x_{i}$ and $\epsilon_{it}$
\end{enumerate}

The likelihood function is given by:
$$ L_{i}(\beta, \gamma, \sigma_{\alpha},\lambda)= \int_{-\infty}^{\infty}
\prod_{t=2}^{T}\Phi(x^{'}_{it}\beta + \gamma y_{it-1} + \lambda y_{i1} +
\tilde{\alpha}) \frac{\phi(\tilde{\alpha})}{\sigma_{\tilde{\alpha}}}
d\tilde{\alpha}$$ 

$$ L(\beta, \gamma, \sigma_{\alpha},\lambda) = \prod_{i=1}^{N}L_{i}(\beta, \gamma, \sigma_{\alpha},\lambda:y_{i},x_{i}$$
The marginal effects are given by:
\begin{equation}
\frac{\delta E[y_{it}|x_{it}, \alpha_{i}]}{\delta x_{it,j}} =
\beta_{j}\phi(x_{it}^{'} \beta + \alpha_{i})
\end{equation}

I report the average marginal effects which are given by: 
$$ \frac{1}{N} \sum_{i}\phi( x_{it}^{'} \beta + \alpha_{i})
\hat{\beta_{j}}$$

\newgeometry{top=4cm}
\begin{center}
\begin{table}[H]
\caption{Dynamic Random Effects Probit (ACF Estimates)}
\label{tab:dynprobitacf}
{
\def\sym#1{\ifmmode^{#1}\else\(^{#1}\)\fi}
\begin{tabular}{l*{4}{c}}
\hline\hline
            &\multicolumn{1}{c}{(1)}&\multicolumn{1}{c}{(2)}&\multicolumn{1}{c}{(3)}&\multicolumn{1}{c}{(4)}\\
            &\multicolumn{1}{c}{exp}&\multicolumn{1}{c}{exp}&\multicolumn{1}{c}{imp}&\multicolumn{1}{c}{imp}\\
\hline
main        &                     &                     &                     &                     \\
lagexp      &       1.834\sym{***}&       1.786\sym{***}&                     &       0.380\sym{***}\\
            &    (0.0258)         &    (0.0260)         &                     &    (0.0233)         \\
[1em]
lpresacf    &       0.216\sym{***}&       0.198\sym{***}&       0.277\sym{***}&       0.266\sym{***}\\
            &    (0.0137)         &    (0.0135)         &    (0.0129)         &    (0.0126)         \\
[1em]
lgfa        &      0.0748\sym{***}&      0.0539\sym{***}&       0.119\sym{***}&       0.110\sym{***}\\
            &    (0.0130)         &    (0.0129)         &    (0.0122)         &    (0.0119)         \\
[1em]
lsalary     &       0.226\sym{***}&       0.201\sym{***}&       0.234\sym{***}&       0.206\sym{***}\\
            &    (0.0137)         &    (0.0136)         &    (0.0124)         &    (0.0121)         \\
[1em]
initexp     &       1.333\sym{***}&       1.264\sym{***}&                     &                     \\
            &    (0.0454)         &    (0.0441)         &                     &                     \\
[1em]
lagimp      &                     &       0.370\sym{***}&       1.600\sym{***}&       1.554\sym{***}\\
            &                     &    (0.0252)         &    (0.0241)         &    (0.0243)         \\
[1em]
initimp     &                     &                     &       1.081\sym{***}&       0.986\sym{***}\\
            &                     &                     &    (0.0383)         &    (0.0370)         \\
[1em]
\_cons      &      -3.264\sym{***}&      -3.122\sym{***}&      -3.655\sym{***}&      -3.617\sym{***}\\
            &     (0.130)         &     (0.128)         &     (0.128)         &     (0.124)         \\
\hline
lnsig2u     &                     &                     &                     &                     \\
\_cons      &      -0.437\sym{***}&      -0.521\sym{***}&      -0.628\sym{***}&      -0.750\sym{***}\\
            &    (0.0612)         &    (0.0623)         &    (0.0590)         &    (0.0613)         \\
\hline
rho         &       0.392         &       0.373         &       0.348         &       0.321         \\
sigma\_u     &       0.804         &       0.771         &       0.731         &       0.687         \\
ll          &    -14513.0         &    -14406.6         &    -15879.7         &    -15749.5         \\
\hline\hline
\multicolumn{5}{l}{\footnotesize Standard errors in parentheses}\\
\multicolumn{5}{l}{\footnotesize \sym{*} \(p<0.05\), \sym{**} \(p<0.01\), \sym{***} \(p<0.001\)}\\
\end{tabular}
}

\end{table}
\end{center}
\restoregeometry
\subsection{Appendix (Dynamic Bivariate Probit Model)}\label{apendix:bivariate}
Let the latent model be: 
$$y^{1*}_{it}= x_{it}^{T}\beta_{1} + y_{it-1}\gamma + \varepsilon_{it}^{1}$$
$$y^{2*}_{it}= x_{it}^{T}\beta_{2} +y_{1t-1}\gamma + \varepsilon_{it}^{2}$$

we have the observed responses as, 
$$y_{it}^{1}= I(y^{1*}_{it}>0)$$
$$y_{it}^{2}= I(y^{2*}_{it}>0)$$

The error distribution is given as follows,
$$(\varepsilon_{it}^{1}, \varepsilon_{it}^{2}) \sim N \begin{pmatrix} 0, & \begin{pmatrix}1 & \rho\\ \rho & 1\end{pmatrix} 
\end{pmatrix}$$

The multivariate normal density  of $f(y^{1*}_{it} , y^{2*}_{it})$  given the assumption above is given by: 
\begin{equation}
\begin{aligned}
    f(y_{it}^{1},y_{it}^{2}) = \frac{1}{2\pi \sigma_{y_{it}^{1}}\sigma_{y_{it}^{2}}\sqrt{1-\rho^2}} exp(\frac{-1}{2(1-\rho^2)} [\frac{(y_{it}^{1} - \mu_{y_{it}^{1}})^2}{\sigma_{y_{it}^{1}}^{2}} + \\
    \frac{(y_{it}^{2} - \mu_{y_{it}^{2}})^2}{\sigma_{y_{it}^{2}}^{2}} 
    -2 \rho \frac{(y_{it}^{1} - \mu_{y_{it}^{1}}) (y_{it}^{2} - \mu_{y_{it}^{2}})
    }{\sigma_{y_{it}^{1}}\sigma_{y_{it}^{2}}}]
  \end{aligned}
\end{equation}
Therefore the likelihood function is given by 

\begin{equation}
\begin{aligned}
L(\beta_1,\beta_2,\gamma_1,\gamma_2) = \Big( \prod
P(Y_{it}^{1}=1,Y_{it}^{2}=1\mid\beta_1,\beta_2,\gamma_1,\gamma_2)^{Y_{it}^{1}Y_{it}^{2}}\\
 P(Y_{it}^{1}=0,Y_{it}^{2}=1\mid\beta_1,\beta_2,\gamma_1,\gamma_2)^{(1-Y_{it}^{1})Y_{it}^{2}}  \\
P(Y_{it}^{1}=1,Y_{it}^{2}=0\mid\beta_1,\beta_2,\gamma_1,\gamma_2)^{Y_{it}^{1}(1-Y_{it}^{2})}\\
P(Y_{it}^{1}=0,Y_{it}^{2}=0\mid\beta_1,\beta_2,\gamma_1,\gamma_2)^{(1-Y_{it}^{1})(1-Y_{it}^{2})} \Big)
\end{aligned}
\end{equation}




% \begin{equation}
% \begin{aligned}
% L(\beta_1,\beta_2) = \sum(\Phi( Y_{1i,t}Y_{2i,t}\ln \Phi(X_1\beta_1,X_2\beta_2,\rho) \\
%  + (1-Y_{1i,t})Y_{2i,t}\ln \Phi(-X_1\beta_1,X_2\beta_2,-\rho) \\
% + Y_{1i,t}(1-Y_{2i,t})\ln \Phi(X_1\beta_1,-X_2\beta_2,-\rho) \\
%  +(1-Y_{1i,t})(1-Y_{2i,t})\ln \Phi(-X_1\beta_1,-X_2\beta_2,\rho) \Big))\\
% \end{aligned}
% \end{equation}
\begin{center}
\begin{table}[H]
\caption{Dynamic Bivariate Probit (ACF Estimates)}
\label{tab:biprobitacf}
\begin{center}
\begin{tabular}{l*{2}{c}}
\hline\hline
            &\multicolumn{1}{c}{(1)}&\multicolumn{1}{c}{(2)}\\
            &\multicolumn{1}{c}{Export
              Decsion}&\multicolumn{1}{c}{Import Decision}\\
\hline\\
$d_{it-1}^{X}$   &       2.539$^{***}$   &       0.397$^{***}$\\
            &        (0.0312)            &    (0.0273)         \\
[1em]                                                          
$d_{it-1}^{M}$   &       0.360$^{***}$   &       2.175$^{***}$\\
            &        (0.0278)            &    (0.0335)         \\
[1em]                                                          
$\hat{\omega}_{it}$&       0.103$^{***}$   &       0.165$^{***}$\\
            &        (0.0152)            &    (0.0150)         \\
[1em]                                                          
$k_{it}$       &       0.0125            &      0.0848$^{***}$\\
            &        (0.0157)            &    (0.0123)         \\
[1em]                                                          
$l_{it}$     &          0.143$^{***}$   &       0.146$^{***}$\\
            &        (0.0173)            &    (0.0173)         \\
[1em]                                                          
Constant      &        -2.212$^{***}$   &      -2.768$^{***}$\\
            &         (0.111)            &    (0.0809)         \\
\hline
\hline
      &                     \\
$corr(\epsilon_{it}^{X},\epsilon_{it}^{M}) $      &       0.439$^{***}$\\
            &    (0.0173)         \\
LR test& $H_{0}:corr(\epsilon_{it}^{X},\epsilon_{it}^{M}) =0$, $\chi^{2}(1)= 641.78$&\\
& p=0.000&\\
Industry Dummies & Yes& \\
Time Dummies& Yes& \\
\hline\hline
\textit{Note:}&\multicolumn{2}{r}{\footnotesize Robust standard errors in parentheses}\\
&\multicolumn{2}{r}{\footnotesize $^{*}$ \(p<0.05\), $^{**}$ \(p<0.01\), $^{***}$ \(p<0.001\)}\\
\end{tabular}
\end{center}

\end{table}
\end{center} 


\end{document}
