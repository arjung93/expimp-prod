 \documentclass[11pt]{article}
\usepackage[font=small,labelfont=bf]{caption}
\usepackage{graphicx}
\usepackage{amsmath}
\usepackage{hyperref}\usepackage{color}
\usepackage{parskip}
\usepackage{float}
\usepackage{tabularx}
\usepackage{appendix}
\usepackage{rotating} \usepackage{verbatim} \usepackage{lscape}
\usepackage{dcolumn} \usepackage{ctable} \usepackage{amssymb}
\usepackage{longtable}
\usepackage{threeparttable}
\usepackage{pdflscape}
\usepackage{cases}

 \usepackage[style=authoryear,backend=biber, url=true]{biblatex}
 \addbibresource{new.bib}
\title{Exporting/Importing and Firm Performance: Evidence from India}
\author{Arjun Gupta (2014592/823125)}
% \institution{Tilburg University}
\newcommand{\alert}[1]{#1}

\newcommand{\floatintro}[1]{
  
  \vspace*{0.1in}
  
  {\footnotesize
    
    #1
    
  }
  
  \vspace*{0.01in}
}
%Introduce floatintros
\definecolor{red}{rgb}{0.0,0.0,0.0}
\hypersetup{colorlinks,breaklinks,linkcolor=red,urlcolor=red,anchorcolor=red,citecolor=red}
\floatstyle{ruled} 
\restylefloat{table} 
\restylefloat{figure}

\begin{document}
\maketitle

\begin{center}
\includegraphics[width=3cm]{./PICS/tilburg.png} 
\end{center}

\begin{abstract}
This research thesis provides empirical evidence whether we observe
learning-by-doing and  self-selection hypothesis amongst Indian
manufacturing firms. The learning-by-doing hypothesis is shown by
endogenously accommodating  the the export/import decision in the
Levinsohn-Petrin productivity estimation procedure. The self-selection phenomenon is
checked with the help of a dynamic random effects probit model. The
complementarity between exporting and importing is shown with the help
of a dynamic bivariate probit model. 
\end{abstract}
\bigskip

\begin{center}
A thesis submitted in partial fulfillment of the requirements for the
degree of Master of Science in Econometrics and Mathematical
Economics\\ 

\bigskip

Tilburg School of Economics and Management\\
Tilburg University
\end{center}
\newpage
\small

\tableofcontents

\newpage

\section{Introduction}\label{sec:introduction}



There is vast empirical literature  that has documented that exporters tend to
out-perform non-exporters  in terms of wages, capital,
productivity. \textcite{bernard1999exceptional}  suggest that this can be due
to the following phenomenon:
\begin{itemize}
\item  Higher productivity leads to  export (Self-selection)
\item Export market participation increases productivity (Learning by doing)
\end{itemize}

Self-selection (SS) hypothesis states that  more productive firms
self-select into export  as  
participation in the trade market is accompanied by additional costs
such as transport costs, establishing a distribution channel,
cost of traversing bureaucratic channels  etc. This means that there
are substantial sunk costs to participating in the trade
market. Therefore, firms which are more productive enter the
export market. 

Learning-by-doing hypothesis for exporting (\textcite{haidar2012trade}) states that exporting firms deal with
tougher competition in the international market as compared to the
domestic market, and therefore must improve their performance to
remain active in the export market. Moreover, participating in the
international market leads knowledge flows from international buyers
to help post entry performance of export starters. This means that exporting should
cause productivity spillovers as well.

However, most of the research in this field has been limited to
exports and research on imports has been relatively low. The same hypothesis (self-selection and learning-by-doing) can  apply
to import behavior of firms. Since importing intermediaries also
involves additional
similar costs like  taxes, transport costs, import duties
etc., firms that are also more productive will \textit{Self-Select}
into entering the import market. Also, since a firm
participating in the import market can have access to higher quality
of intermediary goods, it might lead to lead to productivity benefits
i.e the firms might exhibit \textit{Learning-by-doing} phenomenon. 
\textcite{topalova2011trade}  and \textcite{halpern2011imported}
find that improved access to foreign technology can boost
productivity. 

Since participating in one activity might increase the chances of
participating in other as it might open up new information
channels or because it fosters an increase in productivity and
innovation. The former would decrease the cost of the the other
activity. This means that there might be cost complementarity between
exporting and importing.  

India provides an interesting case as it liberalised its economy in
1992 which resulted in decrease import/export
tariffs, deregulation of markets, reduction of taxes, and greater 
foreign investment. According to \textcite{topalova2011trade}, \textit{the government's trade policy under the Eighth Five-Year Plan (1992-97) ushered
in radical changes to the trade regime by sharply reducing the role of
the import and export control system. The share of products subject to quantitative restrictions
decreased from 87 percent in 1987‐88 to 45 percent in 1994-95, and the actual user
condition on imports was discontinued. And since 1997, the decrease in
output and input tariff has been very marginal.}. Therefore, it is
important to find evidence of productivity gains from participation in
the trade a market and how participation in one activity affects the
participation in the other activity. It is also important to check
whether participation in one trade activity complements participation
in the other activity. This will help us in knowing the effect of one
aspect of the trade market on the other. 

So, my reasearch plan is to investigate:
\begin{itemize}
\item Self-selection hypothesis: Check whether more productive firms
  participate in the expprt/import market
\item Learning-by-doing hypothesis: Check if there are productivity
  spillovers from participation in the export/import market
\item Complementarity between exporting and importing: Check if there
  is complementarity between exporting and importing 
\end{itemize}

\section{Literature Review}
Most papers on this can be differentiated into three different
categories:
\begin{enumerate}
\item Importing and Productivity 
\item Exporting and Producitivity
\item Complementarity between exporting and importing and its effect
  on productivity
\end{enumerate}
In this section, I write down the major literature contributions
towards my topic. 
\subsection{Importing and Productivity}
If a firm resorts to importing inputs, then it can have access to
higher quality of intermediate goods and might pave the way for future
exporting by increasing the productivity or by providing
cost-complementarity between the two activities.  

\textcite{halpern2011imported} studies effect of imports on productivity by estimating a structural
model of importers in a panel of Hungarian firms. They find that imports have
a significant and large effect on firm productivity, about one-half of which is due
to imperfect substitution between foreign and domestic goods. 

\textcite{topalova2011trade} find that the pro-competitive
effects of the tariffs led firms to become more
efficient, the larger impact appears to have come from 
increased access to foreign inputs.
\subsection{Exporting and Productivity}
Most studies find that there is self-selection of more productive
firms into exporting, however there is mixed evidence for
learning-by-exporting. 

\textcite{aw2011}  estimate a dynamic structural model that captures both the behavioral
and technological linkages among R\&D, exporting, and
productivity. They find that Relative to a
plant that does neither activity, export market participation raises future productivity
by 1.96 percent, R\&D investment raises it by 4.79 percent, and undertaking both
activities raises it by 5.56 percent. 

 \textcite{bernard1999exceptional} test the self-selection and
 learning-by-doing hypothesis of exports on firms. They find that Good
 plants become exporters i.e. learn to export. and find that exporting
 increases the survival probability but it does not contribute towards
 productivity growth.  

\textcite{roberts1997decision} quantify the effect of prior exporting
decision on the current exporting decision and test the sunk cost
hypothesis of these activities.  They  develop a dynamic discrete-
 choice model of exporting behavior that separates the roles of profit heterogeneity
 and sunk entry costs in explaining plants' exporting status and find
 that sunk costs are significant as prior export experience increases
 the probability of exporting by 60 percent.  


In terms of work in this field with Indian firm level  data,
\textcite{haidar2012trade} and \textcite{gupta2018exporting} find evidence of
self-selection of more productive firms into exporting, but they do
not find evidence of learning-by-exporting.  
\subsection{Cost Complementarity between Exporting and Importing}
As far as I am aware, there have been two major papers in this field
i.e \textcite{aristei2013firms} and \textcite{kasahara2013productivity}. 

\textcite{aristei2013firms} estimate a bivariate probit model to
understand the two-way relationship between exporting and importing. 
Thy suse  firm-level data for a group of 27 Eastern European and 
Central Asian countries from the World Bank Business Environment 
and Enterprise Performance Survey (BEEPS) over the period 2002–2008, 
and after controlling for size (and other firm-level characteristics),
find that firms’ exporting activity does not increase the
probability of importing, while the latter has a positive effect
 on foreign sales. 

\textcite{kasahara2013productivity} estimate a stochastic model of
exporting and importing that incorporates heterogeneity in transport
costs and estimate export and import complimentarities between the two
trade activities. They find that policies which inhibit the
importation of  foreign intermediates can have a large adverse 
effect on the exportation of final goods.  

Another research paper in this field, \textcite{muuls2009imports} find
that productivity advantage of exports towards non-exporters
may be overstated in the current literature, 
because imports were not taken into account as well as exports.
\subsection{Productivity Estimation}
Productivity estimation in all of the research papers mentioned is done using
the methods highlighted below:
\textcite{olley1992dynamics},\textcite{levinsohn2003estimating},
\textcite{ackerberg2006structural} and \textcite{wooldridge2009estimating}. 
These papers take inspiration from one another and the difference in
estimation mentioned in these papers is very minimal. The method
mentioned in \textcite{olley1992dynamics}
have been explained in section \ref{prodest}

With regards to productivity estimation, \textcite{de2013detecting} highlights the importance of endogenizing
the export decision in the minimisation problem of the productivity
estimation. 

\section{Data}\label{sec:data}
I use annual firm level data from Centre for Monitoring Indian Economy
(CMIE) which provides  data from 1989 to 2017. Table \ref{indicator}
displays the variables I fetch from the database and their meanings. 
 
\begin{center}
\input{./TABLES/indicatordescription.gen}
\end{center}

I chose the variables
which might be the most pertinent to my research question. 
The firm variables stated above are nominal values of Indian Rupees
(Miilion). I fetch Wholseale Price
Index (WPI), which provides the inflation rate of the wholesale prices
and deflate the variables to give real values. Then, I clean the data
to remove missing values and select firms with the broad industry
classification code indicating that they are a manufacturing
firm.
 Table \ref{compnfirms} in the Appendix shows the composition of
firms by year.  
\begin{center}
\input{./TABLES/compnfirms.gen}
\end{center}
 I restrict the time period of the study from 2001 to 2016 as those
 periods have  relatively high number of firms.  Since, firms
are under no legal obligation to report their finances, this might
mean  that small firms are less likely to report their
finances. Therefore, I do not analyse the effect of export/import on
the survival probability of the firm and do not consider the
entry/exit decision of a firm. However, this dataset includes all publicly listed firms as
their firm financials are public information. This might affect the 
results as the dataset is biased towards bigger firms. 

I create two additional variables \textit{Export Value},
\textit{Import Value},
\textit{Domestic Sales}
by adding the following variables from the Table 1.  
\begin{enumerate}
\item Export Value: Sum of the exports of goods and exports of services \textit{sa\_export\_goods $+$ sa\_export\_serv}
\item Import Value: Sum of imports of raw materials, stores and spares,
  finished goods and capital goods \textit{sa\_import\_rawmat $+$        sa\_import\_stores\_spares
  $+$ sa\_import\_fg            $+$sa\_import\_cap\_goods}
\item Domestic Sales: Total Sales - Export Sales \textit{sa\_sales $-$ Export Value}
\end{enumerate}
Then, I create four dummy variables of trade market participation:
\begin{itemize}
\item Export Dummy: $d_{it}^{X}=1$ if \textit{Export Value} $> 0$
\item Import Dummy: $d_{it}^{X}=1$ if \textit{Import Value} $> 0$
\end{itemize} 
This gives four categories on the basis of which we can divide the
firms in the data set: 
\begin{itemize}
\item None: Firms that do not participate in the export/import market
\item Export only: Firms that participate in the export market only
\item Import only: Firms that participate in the import market only
\item Both: Firms that participate in both export/import market
\end{itemize}
Table \ref{comp_table} displays the composition of firms according to their trade market
participation status. It is seen that number of firms that do not
participate in the trade market is low,  around 20 to 35
\%. Surprisingly, the number of firms that participate in the trade
market is really high. Another interesting feature is that the number
of firms that participate only in the import market is higher than the
firms that participate only in the export market. This must mean that
the demand for foreign intermediaries is really high. Also, almost 50 \% of
firms in each year participate in both export/import market. This
might mean that there is a very high proportion of firms that participate
in the both the trade market activities relative to the firms that
participate only in one trade market activity. 
\begin{center}
\input{./TABLES/composition.gen}
\end{center}
\section{Descriptive Statistics}
\subsection{Exporter/Importer Performance}
As a first step to see that firms that participate in the trade market
are better performers than firms that do not participate in the trade market, I calculate the mean and standard
deviation and create the density plots for log of  sales, gross fixed assets,
salaries and  expenditure on power and fuel for firms with different
trade activity status. 
Tables \ref{lsales}, \ref{lgfa}, \ref{tab:lsalary} and
\ref{tab:lrawmat} display the results for the variables mentioned above. 
It can be seen that firms that participate in the trade market have
a distribution that is more skewed towards the right for all the
variables mentioned above. In the case of sales, firms that do not
participate in the trade market have a mean of 5.52 whereas firms that
participate in either export or import have a mean value of 6.86 and
6.82 respectively. Also, firms that participate in both export/import
have higher mean of 7.74.  It is also seen
that firms that participate in the both export/import market have
higher mean gross fixed assets,
salaries, expenditure on power and fuel than firms that participate in
only export and only import. On the other hand, the standard deviation in all
the cases is very similar. This indicates that firms participating
in the trade market has an positive effect on the observable characteristics of the firm.
\begin{center}
\begin{table}[H]
\caption{Summary statistics of Sales (log)}
\label{lsales}
\begin{tabular}{c}
 \includegraphics{./PICS/denslsales.pdf}   \\ 
 \input{./TABLES/sumstatslsales.gen}  \\  
\end{tabular}
\end{table}
\end{center}
\begin{center}
\begin{table}[H]
\caption{Summary statistics of Gross Fixed Assets (log)}
\label{lgfa}
\begin{tabular}{c}
 \includegraphics{./PICS/denslgfa.pdf}   \\ 
 \input{./TABLES/sumstatslgfa.gen}  \\  
\end{tabular}
\end{table}
\end{center}
\begin{center}
\begin{table}[H]
\caption{Summary statistics of Salary (log)}
\label{tab:lsalary}
\begin{tabular}{c}
 \includegraphics{./PICS/denslsalary.pdf}   \\ 
 \input{./TABLES/sumstatslsalary.gen}  \\  
\end{tabular}
\end{table}
\end{center}
\subsection{Trade Premia}
Trade premia is defined as the difference in
attributes of firms based on their trade participation status. I
estimate the trade premia using the following Fixed Effect (FE) regression:
$$ X_{it} = \alpha + \beta_{1} d_{it}^{X}+ \beta_{2} d_{it}^{M}+
\beta_{3} d_{it}^{X}*d_{it}^{M} + \beta_{4} age_{it} + \epsilon_{it}$$
where $X_{it}$ is firm level characteristics such as Sales, Gross
Fixed Assets, Expenditure on Raw Material and Salary, $d_{it}^X$ is
the export dummy,$d_{it}^M$ is
the import  dummy, the interaction term between these two variables
and $age_{it}$ is the age of the firm. I estimate this equation using
fixed-effects regression controlling time fixed
effects. Table \ref{expimppremia} displays the results for the above
regression. 
\begin{center}
\input{./TABLES/expimppremia.gen}
\end{center}
It is seen coefficients for both export and import dummy are positive
and significant at 1\% significance level. This means that firms that
participate in the export/import market have more capital and assets
and spend more on salary and raw materials than firms that do not
participate in the trade market. The interaction term between the
export and import dummy is very low in two cases and not significant
in the other two. This means that firms that participate in the both
export and import have  higher sales,assets etc. than firms that
participate in one of these trade activities. The age variable also
has a positive coefficient and is significant at 1\% significance
level. Therefore, the older a firm becomes the higher its capital,
assets etc. become. 

This section verifies the presence of trade premia in our
dataset. This further substantiates the point that firms that
participate in one trade market activity (export or import) have
better sales, gross fixed assets etc than firms that do not
participate in the trade market. Furthermore, firms that participate
in the both of the trade market activities have a higher trade premia
than the firms that participate in one of the trade activities. 
 In the next section, I descriptively check whether exports have an effect
on imports and vice-versa.  

\subsection{Complementarity between Exporting and Importing}
The difference between density plots of exporting firms that only participate in the
export market and firms that participate in import market as well will help
give a better  idea whether there is an effect of one activity on the other. 
Table \ref{tab:lexport} displays the the density plot  of the log
values of export  for firms that participate only in the
export market and for firms that participate in both export/import
market. 

It is seen in Table \ref{tab:lexport} that firms that participate in both the
export/import market have a higher mean of exports (5.24) than firms that only
participate in the export market (3.56). This suggests that importing has a
positive effect on exporting such that the firms export more than they
would if they did not participate in the import market.     

Table \ref{tab:limport} displays the density plot of the log value of  import  for firms that participate only in the
import market and for firms that participate in both export/import
market. 
It is seen in Table \ref{tab:limport} that firms that participate in both the
export/import market have a higher imports (4.95) than firms that only
participate in the import market (3.41). This suggests that exporting has a
positive effect on importing firms. 

\begin{center}
\begin{table}[H]
\caption{Summary statistics of Export (log)}
\label{tab:lexport}
\begin{tabular}{c}
 \includegraphics{./PICS/denslexport.pdf}   \\ 
 \input{./TABLES/sumstatslexport.gen}  \\  
\end{tabular}
\end{table}
\end{center}

\begin{center}
\begin{table}[H]
\caption{Summary statistics of Import (log)}
\label{tab:limport}
\begin{tabular}{c}
 \includegraphics{./PICS/denslimport.pdf}   \\ 
 \input{./TABLES/sumstatslimport.gen}  \\  
\end{tabular}
\end{table}
\end{center}
Tables \ref{tab:lexport} and \ref{tab:limport} suggest that both these activities have a
positive effect on the other and this might be because importing
complements exporting and vice-versa. Therefore, there might be
complementarity between these
activities that needs further research. 

I estimate the trade premia similar to the regression in the previous
section on exporting value of the discrete
decision to import and vice-versa. This is done by estimating the
following equation using a fixed-effects (FE) regression:

$$  ln(Export)_{it} = \alpha +  \beta_{1} d_{it}^{M}+
+ \beta_{2} age_{it} + \epsilon_{it}$$

$$  ln(Import)_{it} = \alpha + \beta_{1} d_{it}^{X} + \beta_{2} age_{it} + \epsilon_{it}$$ 

\begin{center}
\input{./TABLES/prodpremia.gen}
\end{center}

The first two columns of table \ref{prodpremia} display these
results. It is seen that the discrete decision to import has a positive and
significant effect on the value of imports such that the discrete
decision to participate in the import market increases the value of
export by 0.506 . The discrete decision to
export also has a positive and significant effect on the quantity of
imports by increasing the value of imports by 0.458.  This further suggests the presence of complementarity
between exporting and importing. 

In the next section, I investigate the effect  exporting/importing on
crude measures of productivity. 
\subsection{Productivity and Export/Import}
\textcite{gupta2018exporting} define a rough measure of productivity known
as \textit{capital productivity}. It is defined as the log of value added per
unit of capital used by a firm:

$$ log(VA_{it} - log(K_{it}))$$
where $VA_{it}$ is firm-level value added, computed as total industrial sales plus
change in stock minus power and fuel expenditures, and raw material
expenses. 
 Table \ref{tab:capprod} displays the summary statistics for this variable
based on the trade activity status. It can be seen that mean of capital
productivity increases as activity status moves from \textit{None} to
\textit{Export only/Import only} to \textit{Both}, whereas the
standard deviation also decreases.  
\begin{center}
\begin{table}[H]
\caption{Summary statistics of Capital Productivity (log)}
\label{tab:capprod}
\begin{tabular}{c}
 \includegraphics{./PICS/denscapprod.pdf}   \\ 
 \input{./TABLES/sumstatscapprod.gen}  \\  
\end{tabular}
\end{table}
\end{center}
 It is seen in table \ref{tab:capprod} that firms participating in
the trade market increases the capital productivity  as firms
participating in the trade market have a higher mean than those firms
that do not participate in the trade. Firms that do
not participate in the trade market have a very high standard
deviation of profit to sales.

I also use another rough measure of productivity i.e
\textit{Profit to Sales} which is calculated by dividing the profit of
a firm with its sales. Profit to sales is calculated by dividing
the profit after tax with sales. This measure can be interpreted as a
profitability measure. Table \ref{tab:pts} shows the summary
statistics for this variable. 
\begin{center}
\begin{table}[H]
\caption{Summary statistics of Profit to Sales}
\label{tab:pts}
\begin{tabular}{c}
 \includegraphics{./PICS/denspatsales.pdf}   \\ 
 \input{./TABLES/sumstatspatsales.gen}  \\  
\end{tabular}
\end{table}
\end{center}
The same pattern is observed for these variable as well. The mean of
these values increases from -0.1 when a firm is not participating in
the trade market to -0.03 and  -0.06 when a firm is participating in
import and export respectively to 0.01 when a firm is participating in
both of the trade market activities.  

The last two columns of table \ref{prodpremia} estimate the trade
premia for the crude measures of the productivity defined above i.e
Capital Productivity and Profit-to-sales ratio. It is seen that both
of the crude measures of productivity react positively to the discrete
decisions to import and export. The discrete decision to participate
in the export market increases the capital productivity by 0.159 and
profit to sales ratio by 0.061. The discrete decision to import
increases the  capital productivity by 0.097 and
profit to sales ratio by 0.021. 
Moreover, since the interaction term is really low in one case and 
is not significant in the other.  This suggests that participation in both activities
leads to higher productivity than participation in one activity. 
\subsection{Sunk Cost (Transition Probability)}



Table \ref{tab:transition} displays the transition probabilities
observed in the data  and the values in the brackets represent the
number of firms. It is seen that there are very high levels of persistence from
\textit{None} in t-1 to \textit{None} in t. This means that there must be high sunk
costs to enter in the export/import market since only 12\% of the
firms  that do not participate in the trade market in t-1 start
participating in the trade market in t. Moreover,   high levels of persistence
are also observed in  \textit{Both}  (91.5\%). The levels of
persistence in \textit{Import Only} and \textit{Export Only}  is not
as high. A large portion of firms switch to participating in both the trade market
activities. This might mean that participating in the export in time
period in t-1 complements participating in the import  market in time period
t and vice-versa. Also, the number of firms that flip-flop (i.e switch
trade market status) is low, which provides further evidence that there are fixed costs
to participating in the stock market as well. 
\begin{center}
\begin{table}[H]
\caption{Transition probability}
\label{tab:transition}
\input{./TABLES/transition.gen}
\end{table}
\end{center}


This section provided descriptive evidence that firms that participate
in the trade market are bigger (Trade premia), have higher
productivity than firms that do not participate in the trade
market. It provide evidence that exporting firms export more if they
participate in the import market as well and vice-versa. It also
provided evidence that there is persistence in trade status
(especially when the firms are not participating in the trade market
and when they are participating in the export and import market). 

In the next section, I proceed to calculate productivity using more
sophisticated techniques used in the literature and check the endogenous effect
of the decision to export/import on productivity. Then, check if the
sunk cost hypothesis and self-selection hypothesis is observed using a
dynamic random effects probit model. I end the next section by
estimating a dynamic bivariate probit model to examine the complementary
nature of these two activities. 
\section{ Analysis}
I check if the phenomenon seen
in the descriptive statistics are also observed using more sophisticated
techniques. 

I divide this into three parts and check whether: 
\begin{itemize}
\item Learning-by-Doing: How does lagged choice of export/import
  impact productivity?
\item Self-Selection and Sunk Cost Hypothesis : Selection of more productive firms into
  exporting/importing and persistence of these activities
\item Complementarity between exporting and importing (Lagged and
  Contemporaneous): Does engaging
  in one activity complement participation in the other?
\end{itemize}

\subsection{Learning-by-doing}

 I assume that the firms have a Cobb-Douglus
production function: 

\begin{equation}
y_{it} =   \beta_{l}l_{it} + \beta_{k}K_{it} +
\omega_{it} + \eta_{it} 
\end{equation}
where $l_{it}$ is the labor, $K_{it}$ is capital, $\omega_{it}$ is the
total factor productivity or unobserved productivity and $\eta_{it}$
is a unknown shock affecting the firms output. However, the OLS
estimation provides biased results as it does not account for
simultaneity problem i.e the productivity of a firm should be
correlated with the inputs of production. 

There is a vast literature on the estimation of productivity starting
with the seminal paper in productivity estimation: \textcite{olley1992dynamics} and subsequent
modifications of their method: \textcite{levinsohn2003estimating},
\textcite{ackerberg2006structural} and
\textcite{wooldridge2009estimating}. The estimation strategy of
\textcite{olley1992dynamics} is highlighted in section \ref{op}. 

I use the  methods shown in \textcite{levinsohn2003estimating} and
\textcite{ackerberg2006structural} to estimate Cobb-Douglus
parameters. 


\textcite{levinsohn2003estimating} \textbf{LP} uses a  strategy
similar to \textcite{olley1992dynamics} but use intermediate input demand
as the function to invert out $\omega_{it}$. 
Here, the intermediated material demand function is given by:
$$  m_{it} = m_{it}(\omega_{it}, k_{it})$$
This function is assumed to be montonically increasing and therefore
productivity can be found by inverting the function above. Therefore,
we can write the Cobb-Douglus equation  as: 
$$ y_{it} = \beta_{l}l_{it} + \phi_{it}(m_{it},K_{it})$$
where $\phi_{it}(m_{it},K_{it}) =  \beta_{k}K_{it}+ \omega_{it}(m_{it}, K_{it})$
They suggest the  use of a third degree polynomial to approximate 
$\phi_{it}$ and substitute it into the equation above to give the
following result: 

$$  y_{it} =  \beta_{l}l_{it} + \sum_{i=0}^{3} \sum_{j=0}^{3-i}
\delta_{ij}k_{it}^{i}m_{it}^{j}$$
The coefficient is $\beta_{l}$ is estimtated by OLS using the equation
above as they assume that the labor is a flexible input i.e there are
no labor adjustment costs and $\hat{\phi_{it}}$ is estimtated by
subtracting the effect of labor from
the fitted value of the above equation:
$$ \hat{\phi_{it}} = \hat{y_{it}} - \hat{\beta_{l}}l_{it} =
 \sum_{i=0}^{3} \sum_{j=0}^{3-i}
\hat{\delta_{ij}}k_{it}^{i}m_{it}^{j}$$
Therefore, 
So, for any value of $\beta_{k}^{*}$:
$$\hat{\omega_{it}} = \hat{\phi_{it}} - \beta_{k}^{*}k_{it}$$
$$ y_{it}^{*} = y_{it} - \beta_{l}l_{it} = \beta_{k}K_{it}
+ \omega_{it}(m_{it}, K_{it})$$
Since it is also assumed that $\omega_{it}$ follows a first order markov
process : 
$$\omega_{it} = E[\omega_{it}|\omega_{it-1}] + \epsilon_{it}$$
They also assume a polynomial expansion of the expectation above to give:
$$ \omega_{it}=  \gamma_{o}+\gamma_{1}\omega_{it-1} +
\gamma_{2}\omega_{it-1}^2 + \gamma_{3}\omega_{it-1}^3 + \epsilon_{it} $$ 
Therefore, the value of $\beta_{K}$, for which the expression below is
minimized is chosen to be the coefficient of capital.  
\begin{equation}
\min_{\beta_{k}^{*}}\sum (y_{it} - \hat{\beta_{l}l_{it}} -
\beta_{k}^{*}K_{it} - \hat{E[\omega_{it}|\omega_{it-1}]})^2 
\end{equation}

On the other hand, \textcite{ackerberg2006structural}(ACF) use a
similar strategy but, suggest that labour might be
correlated with productivity. Therefore, they write  the firms input
material demand as a function of productivity, capital as well as
labor : 
$$ m_{it} = f_{it}(\omega_{it}, k_{it}, l_{it})$$
Inverting this function for $\omega_{it}$ and substituting into the
production function results in the following 
equation of the form:
\begin{equation}
y_{it} = \beta_{l}l_{it} + \beta_k k_{it} + f_{it}^{-1}(m_{it},
k_{it}, l_{it})+ \epsilon_{it}
\end{equation}
They suggest that the  labor coefficient along with capital since it is correlated with
productivity. 
They suggest the following steps:
\begin{enumerate}
\item Obtain $\phi_{it}(m_{it}, k_{it}, l_{it}) = \beta_{l}l_{it} + \beta_k k_{it} + f_{it}^{-1}(m_{it},
k_{it}, l_{it})$ by regressing $y_{it}$ on polynomial approximation of
$\phi_{it}(m_{it}, k_{it}, l_{it})$
\item Use the markovian nature of $\omega_{it} =
  E(\omega_{it}|\omega_{it-1}) + e_{it}$
and use the following moment equations to estimate $\beta_{K}$ and
$\beta_{l}$:
\begin{equation}
E[e_{it}|(k_{it}, l_{it-1})]= 0
\end{equation}
\end{enumerate} 

However, exogenously regressing lagged  export/import variables  on
productivity (obtained by getting the residuals) suggests that past
export/import performance does not impact the evolution of
capital/labor. This has been highlighted by \textcite{de2013detecting}, and
they suggest that the trade activities should be accommodated
endogenously in the productivity evolution process. This is done by
accommodating the the lagged export/import variable into the
minimisation procedure of productivity:

$\omega_{it} =
  E(\omega_{it}|\omega_{it-1}, d_{it-1}^{X}, d_{it-1}^{M}) + e_{it}$

% \subsection{Learning-by-doing}


% I estimate the  endogenous  effect of export/import on productivity
% using two techniques  used widely in this field: 
%  \textcite{levinsohn2003estimating} (LP)and
%  \textcite{ackerberg2006structural} (ACF).
% ACF results are shown to highlight the robustness of the results. 
% \subsubsection{Levinsohn-Petri}
% \textcite{levinsohn2003estimating} assume the following  Cobb-Douglus production function: 
% \begin{equation}
% y_{it} = \beta_{o} + \beta_{l}l_{it} + \beta_{k}K_{it} +
% \omega_{it}(m_{it}, k_{it}) + \eta_{it} 
% \end{equation}

% Using OLS residuals of the Cobb-Douglus estimates provide biased
% estimates of productivity since there is correlation between
% productivity and characteristics  of firms.   

I use the log values of the following variables for the estimation
procedure:  gross fixed assets as a measure of capital, salary
as a measure of labor and a expenditure on raw materials as a measure
of intermediated input. Since productivity evolution is  assumed to
have a markovian nature , I assume the following form of productivity
evolution:
$$ \omega_{it} = \alpha_{o} + \alpha_{1}\omega_{it-1} +
\alpha_{2}\omega_{it-1} + \alpha_{3}\omega_{it-1}^{2}+
\alpha_{4}d_{it-1}^{X} + \alpha_{5} d_{it-1}^{M} +
\alpha_{6}d_{it-1}^{X}d_{it-1}^{M}  + \nu_{it}
$$ 
where $d_{it-1}^{X}$ and $d_{it-1}^{M}$ is the discrete decision to
export and import respectively. 

The estimates of the productivity evolution and  the Cobb-Douglus
coefficients  using the Levinsohn-Petrin
method are shown in Table \ref{prod}  and 
Table \ref{regLP} respectively. 
\input{./TABLES/regLP.gen}
It is seen in Table \ref{prod} that productivity evolutions depends strongly
on the past productivity and the coefficients suggest that there is a
non linear relationship with  past
productivity. However, the interesting result is that lagged decision to export does not a have
a significant effect on productivity. However, it is seen that the
discrete decision to import does have a significant effect on productivity i.e the
lagged decision to import causes the productivity to increase by 3 per
cent. The interaction term between exporting and importing also does
not have a significant effect on productivity. 
Table \ref{regLPcont} and \ref{prodcont} display the Cobb-Douglus coefficients when the
productivity evolution is dependent on the lagged continuous value of export
and import rather than the lagged decision to export/import.
and they show results similar to the ones shown in tables \ref{regLP}
and \ref{prod} for the discrete decision. They show that lagged
continuous value of export does not have a significant effect on
productivity, increase in lagged continuous value of import  by 1 unit
increases the productivity by 2.6 \%.  
\input{./TABLES/prod.gen} 



As written before, \textcite{de2013detecting} say that exogenously accommodating the
decision to export/import implies that
past export/import experience has no impact on direct technological
improvements. Therefore, the coefficient of capital will be biased
upwards if the decision to participate in the trade market correlated with capital. Table
\ref{regLPex} displays the coefficients when the export/import
decision is not endogenously accommodated in 
productivity evolution. In this case, the coefficient of capital is
biases upwards (0.452 compared to 0.437 in the endogenous case)  since
the variation in output is attributed more towards capital rather than productivity.   
\input{./TABLES/regLPex.gen}

% \subsubsection{\textcite{ackerberg2006structural}}
The results from the estimation  method mentioned in \textcite{ackerberg2006structural} are shown in \ref{prodACF} and
\ref{regACF}. The coefficient of labor is much lower in this
estimation method which suggests that labor is not a flexible input
which suggests that is also correlated to productivity. However, in terms of the endogenous effect
of export/import on productivity is roughly the same. The export
decision does not have a significant effect on productivity and the
import decision has a 2.6\% increase to the productivity. These
results are similar to the results from the Levinsohn-Petrin
estimation. 
\input{./TABLES/prodACF.gen}
\input{./TABLES/regACF.gen}

The procedure above does not account for immediate impact of the
importing behavior to the output of a firm as it is assumed that
participating in the import market only helps in increasing the
productivity of the firm in the next period. This is a drawback of the
estimation procedure. 


Based on these results above, it is seen that manufacturing firms experience learning-by-importing and
do not experience learning-by-exporting. 
In the next section, I check if more productive self-select into
exporting/importing and if the sunk cost hypothesis is observed for
the discrete decision to export and import with the help of a dynamic probit
random effects model. 
\subsection{Self Selection and Sunk Cost hypothesis}
The self-selection hypothesis states that entry into the trade market
involves fixed and sunk costs and only the most productive firms can
overcome these trade costs. Therefore, to participate in the
export/import market a firm must pay a certain costs and only the most
productive are able to pay the costs. To check this hypothesis, I
estimate a dynamic random effects probit model similar to the model used
in \textcite{roberts1997decision}. I define $d_{it}^{X}$ as the discrete
decision to export, where
 Bellman equation for a profit maximising firm deciding to enter the
 export market is:
\begin{equation}
V_{it}(S_{it})= max_{d_{it}^{X}}E_{t}(\sum_{i=0}^{\infty} \delta^{t+i}R_{i,t+i}|S_{it})
\end{equation}
 where $\delta$ is the one-period discount factor, $S_{it}$ is the
 relevant state variables affecting the firms decision, $R_{ij}$ is
 the revenue. The equation above can also be written as:
\begin{equation}
V_{it}(S_{it})= max_{D_{it}^{X}}(\pi^{D} + d_{it}^{X}(\pi^{X}- f^{X} -
c^{X}(1-d_{it-1}^{X}))  + \sum_{j=t+1}^{\infty} \delta^{t-j}R_{ij}|S_{it})
\end{equation}
where $\pi^{D}$ is the domestic profit, $\pi^{X}$ is the export
profit, $f^{X}$ is the fixed cost of exporting and $c^{X}$ is the sunk
cost of exporting. 
\begin{equation}
V_{it}(S_{it})= max_{D_{it}^{X}}E(\pi^{D} + d_{it}^{X}(\pi^{X}- f^{X} -
c^{X}(1-d_{it-1}^{X})  + \delta E (V_{it}(S_{it+1})|S_{it}, d_{it}^{X})))
\end{equation}

 Thus, a firm will participate in the export market if:
\begin{equation}
\pi_{it}^{*} = \pi^{D}+\pi^{X}  +
\delta E_{t}(V_{i,t+1}(S_{it+1}|S_{it},D_{it}^{X}=1) -
V_{i,t+1}(S_{it+1}|S_{it},D_{it}^{X}=1) ) -  (f^{X} + c^{X}(1-D_{it-1}^{X}))
\end{equation}
It is assumed that the state variables entering the Bellman equation
are the following: $S_{it}= (K_{it}, l_{it}, \omega_{it},
d_{it-1}^{X})$ i.e capital, labor, productivity and lagged decision to
participate in the export market. In the model above, it is assumed
that a firm pays a sunk cost if $d_{it-1}^{X} = 0$. Therefore, if there are no sunk costs, then
according to the Bellman equation above, a firm will export as long as
the the current profits plus the discounted expected value is greater
than the fixed costs of exporting. 
The reduced form expression of the equation above is: 
\begin{equation}
  d_{it}^{X}=\begin{cases}
   1 , & \text{if $\pi_{it}^{*} \geq 0$}.\\
   0 , & \text{if $\pi_{it}^{*}<  0$}.
  \end{cases}
\end{equation}
To enable to reduced form estimation of the model, I write the equation above as a linear
function of the relevant state variables along with dummy variables to
adjust for industry and time effects, to give the equation below:

\begin{equation}
  \pi_{it}^{*}=   \gamma_{1}^{X} d_{it-1}^{X} + 
\gamma_{3}^{X} \hat{\omega}_{it}  + \beta_{1}^{X}K_{it}  +\beta_{2}^{X}L_{it}
IndustrialDummy_{i}^{X} + TimeDummy_{t}^{X}  + \alpha_{i}+ \epsilon_{it}^{X}
\end{equation}
So if the coefficient of  $d_{it-1}^{X}$ is significantly positive, it provides
evidence of sunk cost to participate in the export market as this
means that there is a persistence in the exporting behavior. If the
sunk cost hypothesis did not hold then the probability of exporting
should not depend on the lagged decision to export. And if the
coefficient of $\omega_{t-1}$ is positive, then this proves the
self-selection hypothesis as firms with high
productivity have a higher probability of participating  in the export market.



A similar model can be estimated with the discrete decision to import,
since importing also involves additional fixed and sunk costs, a firm
would be only participate in the import market if gets productivity
benefits to overcome the costs. Learning-from-importing has been
demonstrated in the previous section.  The Bellman equation of a firm
deciding to enter the import market is the following: 

\begin{equation}
V_{it}(S_{it})= max_{D_{it}^{M}}(\pi + d_{it}^{M}(- f^{M} -
c^{M}(1-d_{it-1}^{M})  + \delta E (V_{it}(S_{it+1})|S_{it}, d_{it}^{M})))
\end{equation}
where $\pi$ is total profit of a firm, $f^{M}$ is the fixed cost of
importing and  $c^{M}$ is the sunk cost of importing and  $S_{it} = (K_{it}, l_{it}, \omega_{it},
d_{it-1}^{M})$. 

Since $S_{it}$ contains productivity, this Bellman equation assumes that
importing in time t provides benefits  in t+1 as the decision to
import improves the productivity in the next period. This has been
shown in
the learning-by-doing section. Here, a firm will participate in the
import market if:
\begin{equation}
\pi^{*}= \pi  +
\delta E_{t}(V_{i,t+1}(S_{it+1}|S_{it},d_{it}^{M}=1) -
V_{i,t+1}(S_{it+1}|S_{it},d_{it}^{X}=0) ) -  (f^{M} +
c^{M}(1-d_{it-1}^{M})) > 0
\end{equation}

Therefore,  a firm will import if profit
plus discounted 
productivity benefits outweigh the costs to participate in the import
market. This can be tested with a reduced form dynamic probit model similar to
the discrete decision to export: 
\begin{equation}
  d_{it}^{M}=\begin{cases}
   1 , & \text{if $\pi^{*}  $}.\\
   0 , & \text{otherwise}.
  \end{cases}
\end{equation}
Therefore, the reduced form equations for the \textbf{base model} of both the discrete choices
can be written as:
\begin{equation}
d_{it}^{X*} = \gamma_{1}^{X} d_{it-1}^{X} + 
\gamma_{3}^{X} \hat{\omega}_{it}  + \beta_{1}^{X}K_{it}  +\beta_{2}^{L}L_{it-1}
si_{i}^{X} + \mu_{t}^{X}  + \alpha_{i}+\epsilon_{it}^{X}
\end{equation}
\begin{equation}
d_{it}^{M*} = \gamma_{1}^{M} d_{it-1}^{M} + 
\gamma_{3}^{M} \hat{\omega}_{it}  + \beta_{1}^{M}K_{it}  +\beta_{2}^{L}L_{it}
s_{i}^{M} + \mu_{t}^{M}  + \alpha_{i}+\epsilon_{it}^{M}
\end{equation}
Here $s_{i}$ is a vector of industry dummies and $\mu_{t}$ is a vector
of time dummies. 

To check the presence of lagged complementarity between exporting and
importing, I add another variableto state space: lagged export
choice into the import decision equation and lagged import choice
to export decision equation. Therefore, the Bellman equations that
accounts for lagged complementarity for the export decision is: 
\begin{equation}
V_{it}(S_{it})= max_{D_{it}^{X}}E(\pi^{D} + 
(\lambda *d_{it-1}^{M} + 1 - d_{it-1}^{M})*d_{it}^{X}(\pi^{X}- f^{X} -
c^{X}(1-d_{it-1}^{X})  + \delta E (V_{it}(S_{it+1})|S_{it}, d_{it}^{X})))
\end{equation}
Here, $\lambda$ has a value between 0 and 1 if a firm participated in
the import market in the previous period.

Similarly, the Bellman equation for the import decision is: 
\begin{equation}
V_{it}(S_{it})= max_{D_{it}^{M}}(\pi + (\lambda *d_{it-1}^{M} + 1 - d_{it-1}^{M}) *d_{it}^{M}(- f^{M} -
c^{M}(1-d_{it-1}^{M})  + \delta E (V_{it}(S_{it+1})|S_{it}, d_{it}^{M})))
\end{equation}
Here, $\lambda$ has a value between 0 and 1 if a firm participated in
the export market in the previous period. And the reduced form
equation of the two Bellman equations that account for lagged
complementarity are the following:


\begin{equation}
d_{it}^{X*}=   \gamma_{1}^{X} d_{it-1}^{X} + \gamma_{2}^{M} d_{it-1}^{M}+
\gamma_{3}^{X} \hat{\omega}_{it}  + \beta_{1}^{X}K_{it}  +\beta_{2}^{X}L_{it}+
s_{i}^{X} + \mu_{t}^{X}  + \alpha_{i}+ \epsilon_{it}^{X}
\end{equation}
Here, it is important to notice that  $\gamma_{2}^{M}$ measures the effect of lagged importing on
exporting after accounting for the productivity benefits of importing,
since the equation already uses the productivity
estimates. Therefore, if $\gamma_{2}^{X}>0$, then this means that
importing in time t leads to decrease in cost of exporting at time t$+$1.  
\begin{equation}
d_{it}^{M*}=   \gamma_{1}^{M} d_{it-1}^{M} + \gamma_{2}^{M} d_{it-1}^{M}+
\gamma_{3}^{M} \hat{\omega}_{it}  + \beta_{1}^{M}K_{it}  +\beta_{2}^{M}L_{it}+
s_{i}^{M} + \mu_{t}^{M}  + \alpha_{i}+ \epsilon_{it}^{M}
\end{equation}
In the equation above, if  $\gamma_{2}^{X}>0$, then this means that
exporting in time t leads to decrease in cost of importing at time t$+$1.
  
I use the dynamic random effects probit specification with Wooldridge
method (\textcite{wooldridge2005simple}) which treats the initial conditions problem by accounting for
the correlation of the initial value with $\alpha$:
$$  \alpha_{i}= \gamma d_{i1}+ \tilde{\alpha_{I}} $$
where $ \tilde{\alpha_{I}} \sim N(0, \sigma_{\alpha}^{2}) $

I use the log value of gross fixed assets as a measure of capital, log
value of wages as a measure of labor and use productivity estimates  of
the Levinsohn-Petrin method in the previous section. 

\begin{center}
\begin{table}[H]
\caption{Dynamic Random Effects Probit (Estimates)}
\label{tab:dynprobit}
\begin{center}
\begin{tabular}{l*{4}{c}}
\hline\hline&\multicolumn{2}{c}{Export
              Decision}&\multicolumn{2}{c}{Import Decision}\\
            &\multicolumn{1}{c}{(1)}&\multicolumn{1}{c}{(2)}&\multicolumn{1}{c}{(3)}&\multicolumn{1}{c}{(4)}\\
            &\multicolumn{1}{c}{Base}&\multicolumn{1}{c}{Lagged}&\multicolumn{1}{c}{Base}&\multicolumn{1}{c}{Lagged}\\
&\multicolumn{1}{c}{}&\multicolumn{1}{c}{Complementarity}&\multicolumn{1}{c}{}&\multicolumn{1}{c}{Complementarity}\\
\hline
        &                     &                     &                     &                     \\
$d_{it-1}^{X}$      &       1.834$^{***}$&       1.786$^{***}$&                     &       0.380$^{***}$\\
            &     (71.18)         &     (68.65)         &                           &     (16.31)         \\
[1em]
$d_{it-1}^{M}$      &                     &       0.370$^{***}$&       1.600$^{***}$&       1.554$^{***}$\\
            &                     &     (14.67)         &     (66.32)         &     (63.85)         \\
[1em]
$\hat{\omega}_{it}$       &       0.216$^{***}$&       0.198$^{***}$&       0.277$^{***}$&       0.266$^{***}$\\
            &     (15.83)         &     (14.70)         &     (21.51)         &     (21.16)         \\
[1em]
$K_{it}$        &      0.0669$^{***}$&      0.0467$^{***}$&       0.109$^{***}$&       0.100$^{***}$\\
            &      (5.17)         &      (3.65)         &      (8.99)         &      (8.51)         \\
[1em]
$L_{it}$     &       0.210$^{***}$&       0.186$^{***}$&       0.213$^{***}$&       0.186$^{***}$\\
            &     (15.09)         &     (13.53)         &     (17.06)         &     (15.20)         \\
[1em]
$d_{i1}^{X}$     &       1.333$^{***}$&       1.264$^{***}$&                     &                     \\
            &     (29.35)         &     (28.62)         &                     &                     \\
[1em]
$d_{i1}^{M}$     &                     &                     &       1.081$^{***}$&       0.986$^{***}$\\
            &                     &                     &     (28.27)         &     (26.62)         \\
[1em]
\_cons      &      -3.264$^{***}$&      -3.122$^{***}$&      -3.655$^{***}$&      -3.617$^{***}$\\
            &    (-25.05)         &    (-24.47)         &    (-28.64)         &    (-29.17)         \\
rho         &       0.392         &       0.373         &       0.348
                                    &       0.321         \\
$\sigma^{2}_{\alpha}$     &       0.804$^{***}$         &       0.771$^{***}$         &       0.731$^{***}$
                                    &       0.687$^{***}$         \\
& (0.0245)& (0.023)& (0.021)& (0.021) \\
Log-Likelihood         &    -14513.0         &    -14406.6         &
                                                                     -15879.7
                                    &    -15749.5         \\
\hline
Industry Dummies & Yes& Yes& Yes& Yes\\
Time Dummies& Yes& Yes& Yes& Yes\\
\hline\hline
\multicolumn{5}{l}{\footnotesize \textit{t} statistics in parentheses}\\
\multicolumn{5}{l}{\footnotesize $^{*}$ \(p<0.05\), $^{**}$ \(p<0.01\), $^{***}$ \(p<0.001\)}\\
\end{tabular}
\end{center}

\end{table}
\end{center}

\begin{center}
\begin{table}[H]
\caption{Dynamic Random Effects Probit (Average Marginal Effects)}
\label{tab:dynprobitme}
{
\def\sym#1{\ifmmode^{#1}\else\(^{#1}\)\fi}
\begin{tabular}{l*{4}{c}}
\hline\hline
            &\multicolumn{1}{c}{(1)}&\multicolumn{1}{c}{(2)}&\multicolumn{1}{c}{(3)}&\multicolumn{1}{c}{(4)}\\
            &\multicolumn{1}{c}{} &\multicolumn{1}{c}{} &\multicolumn{1}{c}{} &\multicolumn{1}{c}{} \\
\hline
lagexp      &       0.218\sym{***}&       0.212\sym{***}&                     &      0.0495\sym{***}\\
            &     (53.32)         &     (53.84)         &                     &     (16.34)         \\
[1em]
lpres       &      0.0257\sym{***}&      0.0235\sym{***}&      0.0362\sym{***}&      0.0347\sym{***}\\
            &     (15.99)         &     (14.82)         &     (21.92)         &     (21.55)         \\
[1em]
lgfa        &     0.00797\sym{***}&     0.00553\sym{***}&      0.0142\sym{***}&      0.0130\sym{***}\\
            &      (5.19)         &      (3.66)         &      (9.03)         &      (8.54)         \\
[1em]
lsalary     &      0.0250\sym{***}&      0.0220\sym{***}&      0.0279\sym{***}&      0.0243\sym{***}\\
            &     (15.20)         &     (13.62)         &     (17.27)         &     (15.37)         \\
[1em]
lagimp      &                     &      0.0439\sym{***}&       0.209\sym{***}&       0.202\sym{***}\\
            &                     &     (14.65)         &     (53.75)         &     (53.94)         \\
\hline
\(N\)       &       62485         &       62485         &       62485         &       62485         \\
\hline\hline
\multicolumn{5}{l}{\footnotesize \textit{t} statistics in parentheses}\\
\multicolumn{5}{l}{\footnotesize \sym{*} \(p<0.05\), \sym{**} \(p<0.01\), \sym{***} \(p<0.001\)}\\
\end{tabular}
}

\end{table}
\end{center}

Table \ref{tab:dynprobit} and \ref{tab:dynprobitme} displays the
estimates and average marginal effects of the dynamic random
effects probit model respectively. Columns 1 and 3 display the result for the base
reduced model of importing and exporting, whereas columns 2 and 4
display the estimates when accounting for the lagged complementarity
between exporting and importing.

The likelihood ratio test to see if the base model is rejected by the
model that accounts for lagged complementarity gives the following
results:
\begin{enumerate}
\item Export: \\$H_{o}$: Base Model, $H_{1}$:Model with lagged discrete
  import variable.
$LR= 2[ln(model2) - ln(model1)] = 2[-14406.6 + +14513.0] = 106.4$
The critical value of $\chi^{2}_{1;0.95}= 3.84 $. Therefore $H_{o}$ is
rejected. 
\item Import:\\ $H_{o}$: Base Model, $H_{1}$:Model with lagged discrete
  export variable. 
$LR= 2[ln(model2) - ln(model1)] = 2[-15749.5 + 15879.7 ] = 130.19$
The critical value of $\chi^{2}_{1;0.95}= 3.84 $. Therefore $H_{o}$ is
rejected. 
\end{enumerate}
In both cases, the model that accounts for the lagged complementarity
performs better than the base model. Thereofore, I interpret the model
that acoounts for the lagged complementarity. 
 

 Export: The state dependence parameter is positive
  significant at 1\% level with the highest magnitude amonst all of
  the coefficients. This means that there is persistence in
  exporting behavior which confirms that sunk-cost hypothesis. The average marginal effect of the lagged
  decision to export is 0.218. Productivity also has a significant and
  positive effect on exporting which provides evidence of
  self-selection of high productivity into exporting. Capital and Labor
  also have a positive and significant effect, which tells us that a
  bigger firm has a higher chance of participating in the export
  market. The lagged import is
  significant and positive and this provides evidence  that importing
  at time t increases the probability of exporting at time $t+1$ and
  cofirms the presence of cost complementarity.  The value of
  $sigma^2_{\alpha}$ in the table does not account for the
  contribution of the intial condition. Therefore, the actual 
  unobserved heterogeneity $\sigma^2_{\alpha}= \lambda^2 *
  \hat{Var(d_{i1}^{X}} + \hat{sigma^2_{\tilde{\alpha}}}= 1.16$. This
  means that individual effects capture  about 54\% of the unexplained
  variance. This suggests there are variables other than the ones used
  in the estimation that contribute towards the export market
  participation decision. 


 Import:  The results for the import decision are quite similar to the
 results for the export decision. The state dependence parameter is the most highest
  amongst all variables and
  significant at 1\% level. This means that there is persistence in
  importing behavior and confirms the sunk cost hypothesis for import decision. The average marginal effect of the lagged
  decision to export is 0.202. Productivity is also significant at 1\%
  and has a 
  positive effect on importing. This provides evidence of
  self-selection of high productivity into importing. Capital and Labor
  also have a positive and significant effect, which confirms that
  bigger firms have a higher probability of import participation.  The lagged export coefficient is
  significant and positive and this provides evidence  that exporting
  at time t increases the probability of importing at time $t+1$. The
  lagged cost complementarity hypothesis is also confirmed for the
  import decision.  The
  unobserved heterogeneity is high as it explains about 48\% of the unexplained
  variation in the data and the variance of the unobserved
  heterogeneity is significant. This value is lower than the value
  when the lagged export decision is not included in the dynamic
  specification (Model 1) 

These results provide evidence of the sunk-cost hypothesis,
self-selection of higher productivity and bigger firms into
exporting/importing and lagged complementarity. However, this estimation does not
account for the contemporaneous  complementarity between exporting
and importing. The next section shows the results of a dynamic
bivariate probit model which accounts for simultaneous complementarity
as well. 

% I use the following equations  to verify that more productive firms
% self-select into participating in the export/import market:
% \begin{equation}
% \hat{log(TFP)_{t-j}} = \gamma_{1}log(export)_{it}+ \gamma_{2}log(import)_{it} +
%  \beta c_{i,t-j}
% \end{equation}

% \begin{equation}
% \hat{log(TFP)_{t-j}} = \gamma_{1}d_{it}^{X}+ \gamma_{2}d_{it}^{M} +
% \gamma_{3}d_{it}^{X}d_{it}^{M} + \beta c_{i,t-j}
% \end{equation}
% where $c_{i,t-j}$ contains log of capital and labour. I estimate the
% equation mentioned above for three time periods $j=1,2 \& 3$ and for
% the discrete decision as well as the value of exports/imports.  The
% coefficients are estimated using fixed-effects regression. 
% Table \ref{discprod} and Table \ref{contprod} display  $\gamma$ estimates for equation 2 and 3
% respectively
  
% \input{./TABLES/discprod.gen}
% \input{./TABLES/contprod.gen}
% In the discrete case, productivity of firms which export in year t is
% 12.5 \%, 7\% and 4.2 \% higher than non-exporting firms in in year
% t-1,t-2 and t-3 respectively. And the productivity of firms which
% import in year t is 13.8 \%, 8\% and 4.9 \% higher than non-nonimporting firms in in year
% t-1,t-2 and t-3 respectively. The interaction variables are
% not significant in all the three cases. This suggests that for firms
% to participate in both the markets, their lagged productivity needs to
% be higher than firm who participate in either the export or the export
% market. Another interesting feature is that firms that only import in
% year t have higher lagged productivity than firms that only export in
% year t.

% Tables\ref{discprod} and  \ref{contprod} provide evidence that lagged productivity at for all the three
% time periods before is higher when the firm participates in the export
% market in year t. This gives evidence of self-selection of firms into
% exporting and importing as the lagged productivity for three
% consecutive time periods before exporting/importing has a
% significantly positive value. 
\subsection{Complementarity between exporting and importing(Lagged and
Contemporaneous)}

The bellman equation of a profit maximising firm deciding to export
and import is the following:  

\begin{equation}
V_{it}(S_{it}) = max_d(\pi_{it}^{d} +d_{it}^{X}\pi_{it}^{X} +
F(d_{it}, d_{it-1}) + \beta E(V_{it}(s_{it+1}|s_{it}, d_{it})))
\end{equation}
where $d_{it}= (d_{it}^X, d_{it}^M)$ and $S_{it}= (K_{it}, L_{it},
\omega_{it}, d_{it-1}^X, d_{it-1}^M)$.  $F(d_{it}, d_{it-1})$ is the
fixed/sunk costs the firms must pay to export/import and it takes the
following form:

$F(d_{it}, d_{it-1})= $
\begin{enumerate}
\item   $f^{x} + c^{X}(1 - d_{it-1}^X)$\hfill  for $ (d_{it}^X, d_{it}^M) =
  (1,0) $
\item   $f^{M} + c^{M}(1 - d_{it-1}^M)$\hfill  for $ (d_{it}^X, d_{it}^M) =
  (0,1) $
\item   $\lambda[f^{x} + f^{M} + c^{X}(1 - d_{it-1}^X) + c^{M}(1 -
  d_{it-1}^M)]$  \hfill for $ (d_{it}^X, d_{it}^M) =
  (1,1) $
\item   0  \hfill for $ (d_{it}^X, d_{it}^M) =
  (0,0) $
\end{enumerate}
Here $f^{X}$ is the fixed cost of exporting,$C^{X}$ is the sunk cost
of exporting, $f^{M}$ is the fixed cost of importing, $f^{M}$ is the
fixed cost of importing and $\lambda$ captures the degrees of
complementarity between exporting and importing. 

The reduced form model of the bellman equation is a dynamic bivariate
probit model.  Then, the bivariate dynamic probit model in this takes the following form:

\begin{equation}
  d_{it}^{X}=\begin{cases}
   1 , & \text{if $d_{it}^{X*}>  0$}.\\
   0 , & \text{if $d_{it}^{X*}<  0$}.
  \end{cases}
\end{equation}

\begin{equation}
  d_{it}^{M}=\begin{cases}
   1 , & \text{if $d_{it}^{M*}>  0$}.\\
   0 , & \text{if $d_{it}^{M*}<  0$}.
  \end{cases}
\end{equation}
The discrete decision of exporting and importing is modelled as a function of previous import and
export status controlling for lagged firm characteristics and industry and time fixed
effects. 
\begin{equation}
d_{it}^{X*} = \gamma_{1}^{X} d_{it-1}^{X} + \gamma_{2}^{X} d_{it-1}^{M}+
\gamma_{3}^{X} \hat{\omega}_{it}  + \beta_{1}^{X}K_{it}  +
s_{i}^{X} + \mu_{t}^{X}  + \epsilon_{it}^{X}
\end{equation}
Here $\gamma_{1}$ identifies the state dependence coefficient, $\gamma_{2}$ accounts for
the fact that participating in one activity in time t-1 improves the
odds of participating in the other at time t (lagged complimentarity),$\gamma_{3}$ accounts for
the self-selection mechanism, $\beta_{1}$ accounts for capital at time
t-1 and $s_{i}^{M}$  $\mu_{t}^{M}$ are industrial
and time dummies respectively.

The reduced form specification for import is similar to the export
equation:
\begin{equation}
d_{it}^{M*} = \gamma_{1}^{M} d_{it-1}^{M} + \gamma_{2}^{M} d_{it-1}^{X}+
\gamma_{3}^{M} \hat{\omega}_{it}  + \beta_{1}^{M}K_{it}  +
s_{i}^{M} + \mu_{t}^{M}  + \epsilon_{it}^{M}
\end{equation}

The bivariate specification also allows for the
contemporaneous complementarity between the two choices as
$\epsilon_{it}^{X}$ and $\epsilon_{it}^{M}$ are allowed to be
correlated. This gives gives the following form to the error
structure: 


\[\begin{pmatrix}
\epsilon_{it}^{X} \\
\epsilon_{it}^{M}
\end{pmatrix}\sim N\left(\begin{pmatrix}
0 \\
0
\end{pmatrix},\begin{pmatrix}
1 & \rho \\
\rho & 1
\end{pmatrix}\right)
\]
The estimated $\rho$ measures the correlation between the unobserved
errors between the two activities. Therefore, this provides evidence
of contemporaneous complementarity if it significantly positive after
controlling for different effects. 
This model specification has been used to test the contemporaneous relationship
by \textcite{aristei2013firms} and \textcite{aw2011}. 

This model has a few drawbacks: it does not account for individual
heterogeneity ($\alpha_{i}$) and it does not account for the  initial
conditions problems. Therefore, the model assumes that $d_{i1}$ is 
endogenously given and that the firm characteristics and industry and time  dummy variables account
for the individual heterogeneity between firms.
\begin{center}
\begin{table}[H]
\caption{Dynamic Bivariate Probit (Estimates)}
\label{tab:biprobit}
\begin{center}
\begin{tabular}{l*{2}{c}}
\hline\hline
            &\multicolumn{1}{c}{(1)}&\multicolumn{1}{c}{(2)}\\
            &\multicolumn{1}{c}{Export
              Decsion}&\multicolumn{1}{c}{Import Decision}\\
\hline\\

$d_{it-1}^{X}$  &          2.539$^{***}$    &   0.397$^{***}$ \\
            &    (1    (0.0312)             &(0.0273)         \\
[1em]                                                        
$d_{it-1}^{M}$      &      0.360$^{***}$    &   2.175$^{***}$\\
            &          (0.0278)             &(0.0335)         \\
[1em]                                                        
$\hat{\omega}_{it}$  &     0.103$^{***}$     &  0.165$^{***}$\\
            &          (0.0152)             &(0.0150)         \\
[1em]                                                        
$k_{it}$       &        0.00873              & 0.0788$^{***}$\\
            &          (0.0154)             &(0.0122)         \\
[1em]                                                        
$l_{it}$     &            0.135$^{***}$     &  0.134$^{***}$\\
            &          (0.0182)             &(0.0173)         \\
[1em]                                                        
Constant      &          -2.212$^{***}$     & -2.768$^{***}$\\
            &           (0.111)             &(0.0809)         \\
\hline
$corr(\epsilon_{it}^{X},\epsilon_{it}^{M}) $      &       0.439$^{***}$\\
            &    (0.0173)         \\
LR test& $corr(\epsilon_{it}^{X},\epsilon_{it}^{M})$, $\chi^{2}(1)= 641.056$&\\
& p=0.000&\\
Industry Dummies & Yes& \\
Time Dummies& Yes& \\
\hline\hline
\textit{Note:}&\multicolumn{2}{r}{\footnotesize  Robbust standard errors in parentheses}\\
&\multicolumn{2}{r}{\footnotesize $^{*}$ \(p<0.05\), $^{**}$ \(p<0.01\), $^{***}$ \(p<0.001\)}\\
\end{tabular}
\end{center}

\end{table}
\end{center} 
Table \ref{tab:biprobit} displays the results for dynamic probit specification. All
  the coefficients are significant at 1\% level and are quite similar
  to the coefficients in the previous section.  The
  state-dependence coefficient has the strongest effect amongst all
  the variables, suggesting that there is persistence in both the
  activities and high sunk cost. There is a positive effect of 
  productivity on both activities, providing further evidence of
  self-selection of firms into exporting and importing. The
  coefficients of capital, labor and productivity are positive and
  quite similar to each other. Firms which were importing in the previous year are more
  likely to be exporters and firms which were exporting in the
  previous year are more likely to be importing this year.


The results suggest  simultaneous complementarity between exporters and
  importers as the likelihood-ratio test with the null hypothesis that
  the correlation between the unobserved errors is 0 is rejected at
  1\% significance level. The estimated value of $\rho$ is 0.438 and
  is significantly different than zero. This suggests that there is
  simultaneous complementarity between exporting and importing. 
\section{Conclusion}

The results from this section and descriptive statistics suggest a few
overall themes of the data: 
\begin{itemize}
\item Learning-by-doing: I estimate productivity using
  \textcite{levinsohn2003estimating} such that they are dependent on the
  lagged export and import choices and I get the following results for
  the two trading activities:
\begin{enumerate}
\item Export: Firms do not display learning-by-exporting as the
  coefficient of discrete choice of lagged export decision does not
  have a significant effect on productivity. 
\item Import: Firms display learning-by-exporting a the
  coefficient of discrete choice of lagged import decision does
  have a significant effect on productivity 
\end{enumerate}
\item Self-Selection: I regress the lagged productivity values of the
  estimated productivity for $t= t-1,t-2$ and $t-3$ on the discrete choice
  of exporting and importing after controlling for firm
  characteristics and industry and time fixed effects and get the
  following result:
\begin{enumerate}
\item Export: The coefficient for the discrete export choice has a
  positive effect significant at 1\% level on the lagged values of
  productivity. This suggests that firms learn to export.  
\item Import:  The coefficient for the discrete import choice also has a
  positive effect significant at 1\% level on the lagged values of
  productivity. This suggests that firms learn to import as well. 
\end{enumerate} 
\item Complementarity between exporting and importing: I run a
  bivariate dynamic probit regression of discrete choice of
  exporting/importing on their lagged values, firm characteristics and
  industry and time dummies to get the following results:
\begin{enumerate}
\item Export: There is strong persistence in exporting behavior,
  lagged import decision has a positive effect on current exporting
  behavior. 
\item Import: There is strong persistence in importing behavior,
  lagged export decision has a positive effect on current importing
  behavior. 
\item: Contemporaneous Complementarity: There is a strong presence of
  contemporaneous complementarity since the errors for the equations are
  highly correlated are significant at 1\% level.  
\end{enumerate}
The bivaraite dynamic provide  results suggest that there is strong cost complementarity
between exporting at time t with importing at time t and t-1 and
vice-versa. But, the learning by export result show that there is no
learning by exporting. This must mean that importing must help in
reducing the cost to export since they do not enter the productivity
mechanism. 

\end{itemize}

% \section{Model}

% I use a model inspired from  \textcite{aw2011} and \textcite{kasahara2013productivity}. 

% \subsection{Static Decision}

% A firm i has a standard Cobb-Douglas Production Function and faces a
% marginal cost:

% \begin{equation}
% ln c_{it} = \beta_{o} + \beta_{k}ln k_{it} + \beta_{w}ln w_{t} + \omega_{it}
% \end{equation}
% where 
% \begin{itemize}
% \item $K_{it}$ is the unit of output
% \item $w_{t}$ is a vector of variable input prices common to all firms
% \item $\omega_{it}$ is the productivity shock
% \end{itemize}

% A firm faces a constant elasticity of demand (CES) function assumed to
% be of the Dixit-Stiglitz form :

% \begin{equation}
% Q_{it}^{D} = Q_{t}^{D}(\frac{P_{it}^{D}}{P_{t}^{D}})^{\eta_{d}}= \Phi_{t}^{D} (p_{it}^{D})^{\eta_{d}}
% \end{equation} 
% where $Q_{t}^{d}$ is the industry aggregate output, $P_{t}^{d}$ is
% the price index and $P_{it}^{d}$ is the firm i's price, $\eta_{D}$ is
% the constant elasticity of demand. So, the firms demand depends upon
% aggregate market conditions $\Phi_{t}^{D}$

% The demand function in the export market has a similar structure
% except that it also depends on an industry-specific demand shifter: 
% \begin{equation}
% Q_{it}^{X} =
% Q_{t}^{X}(\frac{P_{it}^{X}}{P_{t}^{X}})^{\eta_{d}}exp(z_{it})= \Phi_{t}^{X} (p_{it}^{X})^{\eta_{d}}exp(z_{it})
% \end{equation} 
% where $z_{it}$ is the unobserved firm specific demand
% shock. 

% Equation 2 can be used to obtain an expression for $P_{it}$ and a
% firms domestic revenue is $R_{it} = P_{it}Q_{it}$, and inserting price
% into the revenue function and taking a log to get the revenue function
% in the domestic market:

% \begin{equation}
% ln r_{it} = (\eta_{d} +1) ln \frac{\eta_{d}}{\eta_{d} +1}  + ln
% \Phi_{t}^{D} + (\eta_{d} +1)(\beta_{k}K_{it} + \beta_{w} ln w_{t} +
% \omega_{it}) 
% \end{equation}
%  The revenue function for the export market can be similarly derived
%  to get:
% \begin{equation}
% ln r_{it} = (\eta_{d} +1) ln \frac{\eta_{d}}{\eta_{d} +1}  + ln
% \Phi_{t}^{X} + (\eta_{d} +1)(\beta_{k}K_{it} + \beta_{w} ln w_{t} +
% \omega_{it})  +  z_{it}
% \end{equation}


% \textcite{das2007market} display a relation between profits and revenue. I
% use this estimate the constant demand of elasticity in both the
% domestic and export market. 
% In the domestic market, the profits are: 
% \begin{equation}
% \pi_{it}^d = \frac{1}{\eta_{d}} r_{it}^{d}(K_{it}, \omega_{it}, \Phi_{t}^{D})
% \end{equation} 

% In the export market, the profits are: 
% \begin{equation}
% \pi_{it}^X = \frac{1}{\eta_{X}} r_{it}^{X}(K_{it}, \omega_{it}, \Phi_{t}^{X})
% \end{equation} 

% \begin{equation}
% tvc_{it} = r_{it}^{D}(1 + \frac{1}{\eta_{D}}) + r_{it}^{D}(1 +
% \frac{1}{\eta_{D}}) + \epsilon_{it}
% \end{equation}


% \subsection{Transition of Productivity}

% The firm-level productivity is allowed to the be endogenously affected
% by the firms decision to export and import just as before. . Therefore, the law of
% motion of productivity is:

% \begin{equation}
% \omega_{it} = g(\omega_{it-1}, d_{it-1}^{X}, d_{it-1}^{M}) + \nu_{it}
% \end{equation}

% \begin{equation}
% \omega_{it} = \alpha_{o} + \alpha_{1}\omega_{it-1} +
% \alpha_{2}\omega_{it-1} + \alpha_{3}\omega_{it-1}^{2}+
% \alpha_{4}d_{it-1}^{X} + \alpha_{5} d_{it-1}^{M} + \alpha_{6}d_{it-1}^{X}d_{it-1}^{M}  \nu_{it}
% \end{equation}

% where $d_{it-1}^{X}$ is an indicator function of the firms lagged export
% participation, $d_{it-1}^{M}$ is an indicator function of the firms lagged import
% participation and $\nu_{it}$ is an iid shock to the productivity. The
% lagged export and import indicator variables allow for
% learning-by-exporting and productivity benefits from importing. 

% The model assumes that productivity is only affected by the intensity
% of export/importing but is only dependent on the decision. 

% The firm-specific demand shock$z_{it}$, and industry index $\Phi_{t}^{X}$  and
% $\Phi_{t}^{X}$ is modelled as an exogenous AR(1) process. 


% \subsection{Dynamic Model}
 
% Firm must pay a fixed/sunk costs of trading. Let $d_{i,t}^X$ be the
% indicator function of participation in the export market and
% $d_{i,t}^M$ be the indicator function of participation in the import
% market. Then the total costs (sunk and fixed) paid by firm i in period t is given by:


% $F(d_{it}, d_{it-1})= $
% \begin{enumerate}
% \item   $f^{x} + c^{X}(1 - d_{it-1}^X)$\hfill  for $ (d_{it}^X, d_{it}^M) =
%   (1,0) $
% \item   $f^{M} + c^{M}(1 - d_{it-1}^M)$\hfill  for $ (d_{it}^X, d_{it}^M) =
%   (0,1) $
% \item   $\lambda[f^{x} + f^{M} + c^{X}(1 - d_{it-1}^X) + c^{M}(1 -
%   d_{it-1}^M)]$  \hfill for $ (d_{it}^X, d_{it}^M) =
%   (1,1) $
% \item   0  \hfill for $ (d_{it}^X, d_{it}^M) =
%   (0,0) $
% \end{enumerate}
% Here $f^{X}$ is the fixed cost of exporting,$C^{X}$ is the sunk cost
% of exporting, $f^{M}$ is the fixed cost of importing, $f^{M}$ is the
% fixed cost of importing and $\lambda$ captures the degrees of
% complementarity between exporting and importing. 

% $$ S_{it} = (\omega_{it}, K_{it}, d_{it-1}^{X}, d_{it-1}^{M},
% \Phi_{t}^{X}, \Phi_{t}^{D}, z_{it})$$

% \begin{equation}
% V_{it}(S_{it}) = max_d(\pi_{it}^{d} +d_{it}\pi_{it}^{X} + F(d_{it}, d_{it-1}) + \beta E(V_{it}(s_{it+1}|s_{it})))
% \end{equation}

% Therefore, for any state vector, the marginal benefit of exporting is
% equal to:

% \begin{equation}
% MBE(S_{it}|d_{it-1}) = \pi_{it}^X + V_{it}(s_{it}|e_{it-1}=1) - V_{it}(s_{it}|e_{it-1}=0)
% \end{equation}

% Therefore, for any state vector, the marginal benefit of importing is
% equal to:

% \begin{equation}
% MBM(S_{it}|d_{it-1}) =  V_{it}(s_{it}|M_{it-1}=1) - V_{it}(s_{it}|M_{it-1}=0)
% \end{equation}

% \begin{equation}
%   V_{it}(s_{it}) = \int (\pi_{it}^D + max_{e_{it}}\{( \pi_{it}^D +
%   e_{it-1}\gamma_{it}^{F} - (1- e_{it})\gamma_{it}^{S}) + V_{it}^{E}(s_{it}) ,
%   V_{it}^{D}(s_{it})\}) dG^{\gamma}
% \end{equation}

 
% \begin{equation}
%   V_{it}(s_{it}) = \int (\pi_{it}^D + max_{e_{it}}\{( \pi_{it}^D +
%   e_{it-1}\gamma_{it}^{F} - (1- e_{it})\gamma_{it}^{S}) + V_{it}^{E}(s_{it}) ,
%   V_{it}^{D}(s_{it})\}) dG^{\gamma}
% \end{equation}


% \begin{equation}
%   V_{it}^E(s_{it}) = \int ( max_{m_{it}}\{ \beta E_{t}
%   V_{it+1}(s_{it+1}|e_{it}=1, m_{it} =1) - m_{it-1} \gamma_{it}^{mf} , \beta E_{t}
%   V_{it+1}(s_{it+1}|e_{it=1}=1, m_{it} =0) \} dG^{\gamma}
% \end{equation}


% \begin{equation}
%   V_{it}^D(s_{it}) = \int ( max_{m_{it}}\{ \beta E_{t}
%   V_{it+1}(s_{it+1}|e_{it}=0, m_{it} =1) - m_{it-1} \gamma_{it}^{mf} , \beta E_{t}
%   V_{it+1}(s_{it+1}|e_{it}=0, m_{it} =0) \} dG^{\gamma}
% \end{equation}

\printbibliography[omitnumbers=false]
\section{Appendix}

\subsection{Appendix (Descriptive Statistics)}
\begin{center}
\begin{table}[H]
\caption{Summary statistics of Expenditure on raw material (log)}
\label{tab:lrawmat}
\begin{tabular}{c}
 \includegraphics{./PICS/denslrawmat.pdf}   \\ 
 \input{./TABLES/sumstatslrawmat.gen}  \\  
\end{tabular}
\end{table}
\end{center}


\begin{center}
\begin{table}[H]
\caption{Summary statistics of expenditure on power and fuel (log)}
\begin{tabular}{c}
 \includegraphics{./PICS/denslpower.pdf}   \\ 
 \input{./TABLES/sumstatslpower.gen}  \\  
\end{tabular}
\end{table}
\end{center}

\subsection{Appendix (Productivity Estimation OP)}\label{prodest}
\label{op}

 \textcite{olley1992dynamics} (OP) use the
following strategy to estimate the Cobb-Douglus function: 


The log transformation of the production function is written as : 
\begin{equation}
y_{it} = \beta_{o} + \beta_{l}l_{it} + \beta_{k}K_{it} + \omega_{it} + \epsilon_{it} 
\end{equation}
where $l_{it}$ is labour, $K_{it}$ is capital, $m_{it}$  is
the demand for materials, $\omega_{it}$ is firm-level-productivity
observable to the firm and $\epsilon_{it}$ are shocks to production.\\
The optimal investment decision of the firm is characterised by the following
function:
\begin{equation}
i_{it}= f_{K}(l_{it-1}, K_{it}, \omega_{it})
\end{equation}
The optimal labor decision is characterised by the following function:
\begin{equation}
l_{it}= f_{K}(l_{it-1}, K_{it}, \omega_{it})
\end{equation}

These are the assumptions made by \textbf{OP}:
\begin{enumerate}
\item $f_{K}(l_{it-1}, K_{it}, \omega_{it})$ is invertible in
  $\omega_{it}$
\item $\omega_{it}$ follows a first order markov process 
\item Investment at period i is not active until period $t+1$:
  $K_{it+1}= (1-\delta)k_{it} + i_{it}$
\item Labor is a perfectly flexible input i.e there are no labor
  adjustment cost
\end{enumerate}

The estimation procedure proceeds in two step:
\begin{enumerate}
\item Estimation of $\beta_{k}$:
Since $\omega_{it}=f_{K}^{-1}(l_{it-1},k_{it},i_{it})$ and inserting
this in the Cobb-Douglus equation to get: 
$$ y_{it} = \beta_{l}l_{it} + \phi_{it}(l_{it-1},i_{it},K_{it})$$
where $\phi_{it}(l_{it-1},i_{it},K_{it}) =  \beta_{k}K_{it}+ f_{K}^{-1}(l_{it-1},k_{it},i_{it})$
$\phi_{it}(l_{it-1},i_{it},K_{it})$ is approximated using a polynomial
expression and $beta_{l}$ is estimated by using OLS on the above
equation
\item Estimation of $\beta_{k}$:
Since $\omega_{it}$ follows a first order markov process, it can be
written as: 
$$ \omega_{it} = h(\omega_{it-1}) + e_{it}$$
Then $\phi_{it}$ can be written as: 
$$ \phi_{it} = \beta_{k}K_{it} + h(\omega_{it-1}) + e_{it}$$
$$ \phi_{it} = \beta_{k}K_{it} + h(\phi_{it-1}- \beta_{k}k_{it-1}) + e_{it}$$
The unknown form of function h is approximated by quadratic function
and for any value of $\beta_K$ to get:
$$ \hat{\phi_{it}} = beta_{k}K_{it} +\gamma_{0}
\gamma_{1}\hat{\omega_{it-1}^{\beta_{k}}}+
\gamma_{2}(\hat{\omega_{it-1}^{\beta_{k}}})^{2}
+ \gamma_{3}(\hat{\omega_{it-1}^{\beta_{k}}})^{3} $$
 This expression is minimised to get the estimate of $\beta_{k}$. 
\end{enumerate}
% \subsubsection{\textcite{levinsohn2003estimating} (LP)}

% \subsubsection{\textcite{ackerberg2006structural}(ACF)}
\subsection{Appendix (Learning-by-doing)}
\input{./TABLES/prodcont.gen} 
\input{./TABLES/regLPcont.gen}

\input{./TABLES/prodACFcont.gen} 
\input{./TABLES/regACFcont.gen}

\subsection{Appendix (Dynamic Random Effects Probit  Model)}
Let i be the unit and t be the time. A dynamic random effects probit
is written as: 
$$ y_{it}^{*} = \gamma y_{i,t-1} + x_{it}'\beta + \alpha_{i} +
\epsilon_{it}; y_{it}=1[y_{it}^{*} > 0]$$
where $\gamma$ is the state dependence parameter.



There are three ways to estimate the above equation: 
\begin{enumerate}
\item Treat $y_{i1}$ as exogenously given and do not explain it
\item Heckman Method
\item Wooldridge Method
\end{enumerate} 
I use the Wooldridge method which assumes that $\alpha_{i}$:
$$\alpha_{i} = \lambda y_{i1} + \tilde{\alpha_{i}}$$

Assumptions: 

\begin{enumerate}
\item The i-units are a random sample
\item $\epsilon_{it} \sim N(0,1)$ is independent of $x_{i}$
\item $\tilde{\alpha_{i}} \sim N(0,\sigma_{alpha}^{2})$ is independent of
  $x_{i}$ and $\epsilon_{it}$
\end{enumerate}

The likelihood function is given by:
$$ L_{i}(\beta, \gamma, \sigma_{\alpha},\lambda)= \int_{-\infty}^{\infty}
\prod_{t=2}^{T}\Phi(x^{'}_{it}\beta + \gamma y_{it-1} + \lambda y_{i1} +
\tilde{\alpha}) \frac{\phi(\tilde{\alpha})}{\sigma_{\tilde{\alpha}}}
d\tilde{\alpha}$$ 

$$ L(\beta, \gamma, \sigma_{\alpha},\lambda) = \prod_{i=1}^{N}L_{i}(\beta, \gamma, \sigma_{\alpha},\lambda:y_{i},x_{i}$$
The marginal effects are given by:
\begin{equation}
\frac{\delta E[y_{it}|x_{it}, \alpha_{i}]}{\delta x_{it,j}} =
\beta_{j}\phi(x_{it}^{'} \beta + \alpha_{i})
\end{equation}

I report the average marginal effects which are given by: 
$$ \frac{1}{N} \sum_{i}\phi( x_{it}^{'} \beta + \alpha_{i})
\hat{\beta_{j}}$$

\subsection{Appendix (Dynamic Bivariate Probit Model)}
Let the latent model be: 
$$y^{*}_{i1,t}= x_{i1,t}^{T}\beta_{1} + y_{i1,t-1}\gamma + \varepsilon_{i1}$$
$$y^{*}_{2ti}= x_{i2t}^{T}\beta_{2}y_{i1,t-1}\gamma + \varepsilon_{i2}$$

we have the observed responses as, 
$$y_{i1,t}= I(y^{*}_{i1,t}>0)$$
$$y_{2i,t}= I(y^{*}_{2i,t}>0)$$

The error distribution is given as follows,
$$(\varepsilon_{i1,t}, \varepsilon_{i2,t}) \sim N \begin{pmatrix} 0, & \begin{pmatrix}1 & \rho\\ \rho & 1\end{pmatrix} 
\end{pmatrix}$$

The multivariate normal density  of $f(y*_{i1,t} , y*_{i2,t})$  given the assumption above is given by: 
\begin{equation}
\begin{aligned}
    f(y_{i1,t},y_{i2}) = \frac{1}{2\pi \sigma_{y_{i1,t}}\sigma_{y_{i1,t}}\sqrt{1-\rho^2}} exp(\frac{-1}{2(1-\rho^2)} [\frac{(y_{i1,t} - \mu_{y_{i1,t}})^2}{\sigma_{y_{i1,t}}^{2}} + \\
    \frac{(y_{i2} - \mu_{y_{i2}})^2}{\sigma_{y_{i2}}^{2}} 
    -2 \rho \frac{(y_{i1,t} - \mu_{y_{i1,t}}) (y_{i2} - \mu_{y_{i2}})
    }{\sigma_{y_{i1,t}}\sigma_{y_{i2}}}]
  \end{aligned}
\end{equation}
Therefore the log-likelihood function is given by 

\begin{equation}
\begin{aligned}
L(\beta_1,\beta_2,\gamma_1,\gamma_2) = \Big( \prod
P(Y_{1i,t}=1,Y_{2i,t}=1\mid\beta_1,\beta_2,\gamma_1,\gamma_2)^{Y_{1i,t}Y_{2i,t}}\\
 P(Y_{1i,t}=0,Y_{2i,t}=1\mid\beta_1,\beta_2,\gamma_1,\gamma_2)^{(1-Y_{1i,t})Y_{2i,t}}  \\
P(Y_{1i,t}=1,Y_{2i,t}=0\mid\beta_1,\beta_2,\gamma_1,\gamma_2)^{Y_{1i,t}(1-Y_{2i,t})}\\
P(Y_{1i,t}=0,Y_{2i,t}=0\mid\beta_1,\beta_2,\gamma_1,\gamma_2)^{(1-Y_{1i,t})(1-Y_{2i,t})} \Big)
\end{aligned}
\end{equation}




\begin{equation}
\begin{aligned}
L(\beta_1,\beta_2) = \sum(\Phi( Y_{1i,t}Y_{2i,t}\ln \Phi(X_1\beta_1,X_2\beta_2,\rho) \\
 + (1-Y_{1i,t})Y_{2i,t}\ln \Phi(-X_1\beta_1,X_2\beta_2,-\rho) \\
+ Y_{1i,t}(1-Y_{2i,t})\ln \Phi(X_1\beta_1,-X_2\beta_2,-\rho) \\
 +(1-Y_{1i,t})(1-Y_{2i,t})\ln \Phi(-X_1\beta_1,-X_2\beta_2,\rho) \Big))\\
\end{aligned}
\end{equation}


\end{document}
